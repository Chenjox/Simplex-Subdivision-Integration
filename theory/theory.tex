% !TeX root: .\Beleg5.tex
% arara: clean: { files: [theory.log,theory.aux,theory.thm,theory.slnc] }
% arara: pdflatex
% arara: pdflatex
% arara: biber
% arara: pdflatex
\documentclass{mitschrift}
\usepackage[utf8]{inputenc}
\usepackage[T1]{fontenc}
\usepackage[ngerman]{babel}

%\usepackage{isodate}
\usepackage{graphicx}
\usepackage{tikz}
\usetikzlibrary{calc}
\usetikzlibrary{decorations.markings}
\usepackage[style=numeric,sorting=none]{biblatex}

\addbibresource{sources.bib}

\newcommand{\R}{\mathbb{R}}
\newcommand{\Bary}{\mathbb{B}}

\doctitel{Simplex Integration}
\docuntertitel{Trials and Errors for transcendental Integrals}

\tikzset{
  engineering axonometry/.style={x={(222:0.5cm)}, y={(-7:1cm)}, z={(90:1cm)}}
}
\tikzset{->-/.style={decoration={
  markings,
  mark=at position #1 with {\arrow{>}}},postaction={decorate}},
  ->-/.default=0.5
}

\newcommand{\simplex}{\boldsymbol{\Delta}}

\begin{document}
\makekopf

\tableofcontents

\pagebreak

\part{Theory}

\section{Problem}

Total Energy in the System: \begin{equation}
    \Pi = \int_\Omega g(\phi) \psi(\boldsymbol{u}) \diff \Omega + \frac{G_c}{2l} \int_\Omega \phi^2 + l^2 \nabla \phi \cdot \nabla \phi \diff \Omega \abbild \min
\end{equation} 

Degradation Function \begin{equation}
    g(\phi) = \left(1 - \phi^2\right) + k
\end{equation} mit \begin{conditions}
    k  & being a small but finite scalar such as $\num{10e-6}$ \\
    G_c & critical energy release rate, material parameter \\
    l  & \emph{width} of phase field \\
\psi (u) & strain energy density function\\
    u & displacement function \\
    \phi & phase field parameter, ansatz function discussed below \\
    \nabla \phi & gradient of phase field parameter \\
\end{conditions}

\begin{equation}
    \vdiff_u \Pi = \int_\Omega g(\phi) \sigma(\boldsymbol{u}) \pardiff{\varepsilon}{\boldsymbol{u}} \vdiff \boldsymbol{u} = 0
\end{equation}

\begin{equation}
    \vdiff_\phi \Pi = \int_\Omega 2(\phi - 1) \vdiff\phi \psi(\boldsymbol{u}) \diff \Omega + \frac{G_c}{l} \int_\Omega \phi \vdiff \phi + l^2 \nabla \phi \cdot \nabla \vdiff \phi \diff \Omega = 0
\end{equation}

\subsection{Ansatz functions}

\begin{equation}
    \boldsymbol{u} = \sum_i N_i \boldsymbol{u}_i + \sum_i N_i F \boldsymbol{a}_i 
\end{equation} mit \begin{conditions}
    N_i & are quadratic lagrange (standard) shape functions for tetrahedrons \\
    \boldsymbol{U}_i = \boldsymbol{u}_i, \boldsymbol{a}_i & are nodal degrees of freedom for displacement function \\
    F & is an enrichment function (sigmoid like, depends on $\phi$, later) \\
\end{conditions}

\begin{equation}
    f_{\text{base}} = \sum_i N_i \phi_i
\end{equation}

\begin{equation}
    \varsigmait = \frac{f_{\text{base}}}{\sqrt[4]{f_{\text{base}}^2 + k_{\text{res}}}}
\end{equation} mit \begin{conditions}
k_\text{reg} & small but finite parameter \\
\end{conditions}

\begin{align}
    \phi &= \exp(-\varsigmait) \\
    \phi &= \exp(-\frac{\varsigmait}{l})
\end{align}

we need to be able to integrate the residual vectors and the stiffness matrices
efficiently and accurately
\begin{equation}
    \vdiff_{\boldsymbol{U}_i} \Pi = \int_\Omega g(\phi)\boldsymbol{\sigma}(\boldsymbol{u})\pardiff{\varepsilon}{\boldsymbol{u}}\cdot \pardiff{\boldsymbol{u}}{\boldsymbol{U}} \diff \Omega \cdot \vdiff \boldsymbol{U}_i
\end{equation}

\begin{equation}
    \vdiff_{\phi_i} \Pi = \int_\Omega 2(\phi - 1) \pardiff{\phi}{\phi_i} \psi(\boldsymbol{u}) \diff \Omega \vdiff\phi_i + \frac{G_c}{l} \int_\Omega \phi \pardiff{\phi}{\phi_i} + l^2 \nabla \phi \cdot \pardiff{\nabla \phi}{\phi_i} \diff \Omega \vdiff \phi_i = 0
\end{equation}

\begin{equation}
    \Delta_{U_i} \vdiff_{U_i} \Pi = U_i \cdot \int_\Omega g(\phi) \pardiff{\varepsilon}{U_i}\cdot \mathbb{C}\cdot \pardiff{\varepsilon}{U_i} \diff \Omega \cdot \vdiff U_i 
\end{equation}

\begin{align}
    \Delta_{\phi_j} \vdiff_{\phi_i} \Pi &= \phi_j \int_\Omega 2\left(\pardiff{\phi}{\phi_i}\right)^2 \psi(u) \diff \Omega \vdiff \phi_i 
    + \phi_j \int_\Omega 2(\phi -1 )\pardiff[^2]{\phi}{\phi_i^2} \psi(u) \diff \Omega \vdiff \phi_i \\
&\phantom{={}} +\phi_j \frac{G_c}{l} \int_\Omega \pardiff{\phi}{\phi_i} + \pardiff[^2]{\phi}{\phi_i^2} + l^2 \pardiff{\nabla \phi}{\phi_j} \cdot \pardiff{\nabla \phi}{\phi_i} + l^2 \nabla \phi \cdot \pardiff[^2]{\nabla \phi}{\phi_i^2} \diff \Omega \vdiff \phi_i
\end{align}

\subsection{Numerical Work}

\begin{align} % https://www.desmos.com/calculator/rcnkmh4bgf
    \pardiff{\varsigmait}{f_{\text{base}}} &= \biggl(1 - \frac{f_{\text{base}}^2}{2\bigl(f_{\text{base}}^2 + k_{\text{res}}\bigr)}\biggr) \cdot \frac{1}{\sqrt[4]{f_{\text{base}}^2 + k_{\text{res}}}}\\
    \pardiff[^2]{\varsigmait}{f_{\text{base}}^2} &= \biggl(\frac{5f_{\text{base}}^3}{\bigl(f_{\text{base}}^2 + k_{\text{res}}\bigr)} - 6 f_{\text{base}}\biggr) \cdot \frac{1}{4 {\sqrt[4]{f_{\text{base}}^2 + k_{\text{res}}}}^5}\\
    \pardiff[^3]{\varsigmait}{f_{\text{base}}^3}&= \frac{3\left(-4k_{\text{res}}^{2}+12k_{\text{res}}f_{\text{base}}^{2}+f_{\text{base}}^{4}\right)}{2\left(f_{\text{base}}^{2}+k_{\text{res}}\right)^{2}}\cdot\frac{1}{4{\sqrt[4]{f_{\text{base}}^2 + k_{\text{res}}}}^5}
\end{align}

\subsection{Setting up the Integrals in Question}

The Vectors in Question are:
\begin{align}
    \boldsymbol{N} &= \\
    \boldsymbol{B} &= 
\end{align}

\begin{align}
    \pardiff{\phi}{\phi_i} &= \pardiff{\phi}{\varsigmait} \cdot \pardiff{\varsigmait}{f_{\text{base}}} \cdot \frac{f_{\text{base}}}{\phi_i} \\
    &= -\frac{1}{l} \cdot \phi \cdot \pardiff{\varsigmait}{f_{\text{base}}} \cdot N_i \\
    \pardiff[^2]{\phi}{\phi_i^2} &= \pardiff{}{\phi_i} \Bigl(  -\frac{1}{l} \cdot \phi \cdot \pardiff{\varsigmait}{f_{\text{base}}} \cdot N_i  \Bigr)\\
    &= -\frac{1}{l} \pardiff{}{\phi_i} \Bigl( \phi \cdot \pardiff{\varsigmait}{f_{\text{base}}} \Bigr)\cdot N_i \\
    &= -\frac{1}{l} \Bigl( \pardiff{\phi}{\phi_i} \cdot \pardiff{\varsigmait}{f_{\text{base}}} + \phi \cdot \pardiff{\varsigmait}{f_{\text{base}} \pdiff \phi} \Bigr)\cdot N_i \\
    &= -\frac{1}{l} \Bigl( \pardiff{\phi}{\phi_i} \cdot \pardiff{\varsigmait}{f_{\text{base}}} + \phi \cdot \pardiff{\varsigmait}{f_{\text{base}} \pdiff \phi} \Bigr)\cdot N_i \\
    &= -\frac{1}{l} \Bigl( \pardiff{\phi}{\phi_i} \cdot \pardiff{\varsigmait}{f_{\text{base}}} + \phi \cdot \pardiff[^2]{\varsigmait}{f_{\text{base}}^2} \pardiff{f_{\text{base}}}{\phi} \Bigr)\cdot N_i \\
    &= -\frac{1}{l} \Bigl( -\frac{1}{l} \cdot \phi \cdot \Bigl(\pardiff{\varsigmait}{f_{\text{base}}}\Bigr)^2 \cdot N_i + \phi \cdot \pardiff[^2]{\varsigmait}{f_{\text{base}}^2} N_i \Bigr)\cdot N_i \\
    &= \frac{1}{l^2}\phi \Bigl( \Bigl(\pardiff{\varsigmait}{f_{\text{base}}}\Bigr)^2 - l \cdot \pardiff[^2]{\varsigmait}{f_{\text{base}}^2} \Bigr)\cdot N_i^2 \\
    \pardiff[^2]{\phi}{\phi_i^2} &= \frac{1}{l^2} \biggl( \biggl(\pardiff{\varsigmait}{f_{\text{base}}}\biggr)^2 - \pardiff[^2]{\varsigmait}{f_{\text{base}}^2} \biggr) N_i \cdot N_j \\
\end{align}

\pagebreak

\section{Simplex Integration in $n=2$}


First start with definitions:

\paragraph{Pure Integration Strategy} is any quadrature formula of the simplex: \begin{equation}
    I = \iint_{\simplex} f(\xi_1, \xi_2, \xi_3) \diff \simplex \approx \sum_{i} w_i f(\xi_{1,i}, \xi_{2,i}, \xi_{3,i})
\end{equation} The Term \emph{pure} is used for telling them apart from subdivision integrators.

\paragraph{Subdivision Integration Strategy} are Integrators of the form: \begin{equation}
    I = \iint_{\simplex} f(\xi_1, \xi_2, \xi_3) \diff \simplex = \sum_{i} \iint_{\simplex_i} f(\xi_1, \xi_2, \xi_3) \diff \simplex_i
\end{equation} which then will be evaluated by pure Integrators.

\subsection{Element Description}

\begin{marginfigure}
    \centering
    \begin{tikzpicture}
        \draw[black,thick] (0,0) -- (-60:3cm) -- (-120:3cm) -- cycle;
        \path (0,0) node[above] {$1$};
        \path (-60:3cm) node[below right] {$2$};
        \path (-120:3cm) node[below left] {$3$};
        \path (-60:1.5cm) node[above right] {$5$};
        \path (-120:1.5cm) node[above left] {$4$};
        \path (-60:3cm) -- (-120:3cm) node[midway,below] {$6$};
    \end{tikzpicture}
\end{marginfigure}

Shape Functions in barycentric coordinates \begin{align}
    N_1(\xi_1,\xi_2,\xi_3) &= \xi_1 \\
    N_2(\xi_1,\xi_2,\xi_3) &= \xi_2 \\
    N_3(\xi_1,\xi_2,\xi_3) &= \xi_3 \\
    N_4(\xi_1,\xi_2,\xi_3) &= 4 \xi_1\xi_3 \\
    N_5(\xi_1,\xi_2,\xi_3) &= 4 \xi_1\xi_2 \\
    N_6(\xi_1,\xi_2,\xi_3) &= 4 \xi_2\xi_3
\end{align}

Barycentric Interpolation Formula $P: \mathbb{B}^3 \abbild \mathbb{R}^2$ \begin{align}
    P(\xi_1,\xi_2,\xi_3) &= p_1 \xi_1 + p_2 \xi_2 + p_3 \xi_3
\end{align} mit $p_i \in \mathbb{R}^2$, $\xi_i \in [0,1]$

$\xi$-$\eta$-Transformation \begin{align}
    \xi_1 &:=  1 - \xi - \eta  \\
    \xi_2 &:= \xi\\
    \xi_3 &:=  \eta
\end{align} mit $\xi \in [0,1],\, \eta \in [0,1]$

Es gilt: \begin{align}
    T(\xi, \eta) &= \begin{bmatrix}
        1 - \xi - \eta \\
        \xi \\
        \eta \\
    \end{bmatrix} \\
    T^{-1}(\xi_1,\xi_2,\xi_3) &= \xi_1 \begin{bmatrix}
        0\\
        0\\
    \end{bmatrix} + \xi_2 \begin{bmatrix}
        1\\
        0\\
    \end{bmatrix} + \xi_3 \begin{bmatrix}
        0\\
        1\\
    \end{bmatrix}
\end{align}

\subsection{Pure Integration Strategies}

\subsubsection{Quadrilaterial Integrator}

Characteristic points in barycentric coordinates
\begin{marginfigure}
    \centering
    \begin{tikzpicture}
        \draw[black,thick] (0,0) coordinate (ca) 
          -- (-60:3cm) coordinate (cb) 
          -- (-120:3cm) coordinate (cc) -- cycle;
        %\path (0,0) node[above] {$1$};
        %\path (-60:3cm) node[above right] {$2$};
        %\path (-120:3cm) node[above left] {$3$};
        %\path (-60:1.5cm) node[above right] {$5$};
        %\path (-120:1.5cm) node[above left] {$4$};
        %\path (-60:3cm) -- (-120:3cm) node[midway,above] {$6$};
        \draw[fill] (barycentric cs:ca=0.333,cb=0.333,cc=0.333) coordinate (cbb) circle (2pt) node[below right] {$B_C$};
        \draw[fill] (barycentric cs:ca=1.0,cb=0.0,cc=0.0) coordinate (cb1) circle (2pt) node[above] {$B_1$};
        \draw[fill] (barycentric cs:ca=0.0,cb=1.0,cc=0.0) coordinate (cb2) circle (2pt) node[above right] {$B_2$};
        \draw[fill] (barycentric cs:ca=0.0,cb=0.0,cc=1.0) coordinate (cb3) circle (2pt) node[above left] {$B_3$};
        \draw[fill] (barycentric cs:ca=0.5,cb=0.0,cc=0.5) coordinate (cb4) circle (2pt) node[above left] {$B_4$};
        \draw[fill] (barycentric cs:ca=0.5,cb=0.5,cc=0.0) coordinate (cb5) circle (2pt) node[above right] {$B_5$};
        \draw[fill] (barycentric cs:ca=0.0,cb=0.5,cc=0.5) coordinate (cb6) circle (2pt) node[below] {$B_6$};
        \draw[blue,semithick] (cb1) -- (cb4) -- (cbb) -- (cb5) -- cycle;
        \draw[blue,semithick] (cb2) -- (cb6) -- (cbb) -- (cb5) -- cycle;
        \draw[blue,semithick] (cb3) -- (cb4) -- (cbb) -- (cb6) -- cycle;
        \node at (barycentric cs:cb1=2,cb4=1,cbb=1,cb5=1) {$D_1$};
        \node at (barycentric cs:cb2=2,cb6=1,cbb=1,cb5=1) {$D_2$};
        \node at (barycentric cs:cb3=2,cb4=1,cbb=1,cb6=1) {$D_3$};
    \end{tikzpicture}
\end{marginfigure}
\begin{align}
    B_C &= \begin{bmatrix}
        \frac{1}{3} \\
        \frac{1}{3} \\
        \frac{1}{3} \\
    \end{bmatrix} \label{eqn:characteristicPoint1}\\ 
    B_1 &= \begin{bmatrix}
        1 \\
        0 \\
        0 \\
    \end{bmatrix}& B_2 &= \begin{bmatrix}
        0 \\
        1 \\
        0 \\
    \end{bmatrix}& B_3 &= \begin{bmatrix}
        0 \\
        0 \\
        1 \\
    \end{bmatrix} \label{eqn:characteristicPoint2}\\
    B_4 &= \begin{bmatrix}
        0.5 \\
        0 \\
        0.5 \\
    \end{bmatrix}& 
    B_5 &= \begin{bmatrix}
        0.5 \\
        0.5 \\
        0 \\
    \end{bmatrix}& 
    B_6 &= \begin{bmatrix}
        0 \\
        0.5 \\
        0.5 \\
    \end{bmatrix} \label{eqn:characteristicPoint3}
\end{align}

Domain of a Simplex $\boldsymbol{\Delta}$ can be decomposed into three disjunct subdomains: \begin{equation}
    \boldsymbol{\Delta} = D_1 \cup D_2 \cup D_3
\end{equation}
Therefore the double-Integral \begin{equation}
    \iint_{\boldsymbol{\Delta}} F \diff \boldsymbol{\Delta} = \iint_{D_1} F \diff D_1 + \iint_{D_2} F \diff D_2 + \iint_{D_3} F \diff D_3
\end{equation}

Mapping functions from the $[-1,1] \times [-1,1]$ $X$-$Y$-Unit Square \begin{align}
    g_{1}(X) &= \frac{X}{2} + \frac{1}{2} & g_2(X) &= -\frac{X}{2} + \frac{1}{2} \\
    \pardiff{g_1}{X} &= \frac{1}{2} & \pardiff{g_2}{X} &= -\frac{1}{2} \\
    g_{1}(Y) &= \frac{Y}{2} + \frac{1}{2} & g_2(Y) &= -\frac{Y}{2} + \frac{1}{2} \\
    \pardiff{g_1}{Y} &= \frac{1}{2} & \pardiff{g_2}{Y} &= -\frac{1}{2} \\
    G_1(X,Y) &= g_1(X)g_1(Y) & G_2(X,Y) &= g_1(X)g_2(Y) \\
    G_3(X,Y) &= g_2(X)g_1(Y) & G_4(X,Y) &= g_2(X)g_2(Y)
\end{align} to barycentric coordinates of the $D_1,\, D_2,\, D_3$ Quadrilaterials \begin{align}
    B_{D_1}(X,Y) &= B_1 \cdot G_1(X,Y) + B_5 \cdot G_2(X,Y) + B_4 \cdot G_3(X,Y) + B_C \cdot G_4(X,Y) \label{eqn:DomainTrans1} \\
    B_{D_2}(X,Y) &= B_2 \cdot G_1(X,Y) + B_6 \cdot G_2(X,Y) + B_5 \cdot G_3(X,Y) + B_C \cdot G_4(X,Y) \label{eqn:DomainTrans2} \\
    B_{D_3}(X,Y) &= B_3 \cdot G_1(X,Y) + B_4 \cdot G_2(X,Y) + B_6 \cdot G_3(X,Y) + B_C \cdot G_4(X,Y) \label{eqn:DomainTrans3}
\end{align}

\paragraph{Numerical Integration Scheme}

The Integration is done on the Square $[-1,1] \times [-1, 1]$, which allows for Gaussian Integration to be used: \begin{equation}
    \iint_{[-1,1] \times [-1, 1]} F(X,Y) \diff (X,Y) \approx \sum_i \sum_j F(X_i,X_j) w_i w_j 
\end{equation}
The Gauss-Points $(X_i, X_j)$ and their weights $w_i,w_j$ on the Square can be deduced from the one dimensional Gaussian Integration \begin{equation}
    \int_{-1}^1 H(X) \diff X \approx \sum_i H(X_i) w_i
\end{equation}

The Weights and Points of the 1D Gauss-Legendre Integration are given as: \begin{align*}
    n &= 1 & X &= 0 & w &= 2 \\
    n &= 2 & X &= \sqrt{\frac{1}{3}} & w &= 1 \\
    & & X &= -\sqrt{\frac{1}{3}} & w &= 1 \\
    n &= 3 & X &= \sqrt{\frac{3}{5}} & w &= \frac{5}{9} \\
    & & X &= 0 & w &= \frac{8}{9} \\
    & & X &= -\sqrt{\frac{3}{5}} & w &= \frac{5}{9} \\
\end{align*}

\paragraph{Integral transformation} from the 3 Domains into any simplex.

\begin{Figure}
    \centering
    \begin{tikzpicture}
        \begin{scope}
          \node at (2.5,-0.5) {Euclidean Space $\mathbb{R}^2$};
          \draw[-latex] (0,0) -- (1,0) node[below] {$x$};
          \draw[-latex] (0,0) -- (0,1) node[left] {$y$};
          \draw[black,thick] (3.5,1) coordinate (ca) 
          -- (3,4) coordinate (cb) 
          -- (1,1) coordinate (cc) -- cycle;
          %\path (0,0) node[above] {$1$};
          %\path (-60:3cm) node[above right] {$2$};
          %\path (-120:3cm) node[above left] {$3$};
          %\path (-60:1.5cm) node[above right] {$5$};
          %\path (-120:1.5cm) node[above left] {$4$};
          %\path (-60:3cm) -- (-120:3cm) node[midway,above] {$6$};
          %\draw[fill] (barycentric cs:ca=0.333,cb=0.333,cc=0.333) coordinate (cbb) circle (2pt) node[below right] {$B_C$};
          \draw[fill] (barycentric cs:ca=0.333,cb=0.333,cc=0.333) coordinate (cbb) circle (2pt); %node[below right] {$(x_B,y_B)$};
          \draw[fill] (barycentric cs:ca=1.0,cb=0.0,cc=0.0) coordinate (cb1) circle (2pt) node[below]             {$(x_1,y_1)$};
          \draw[fill] (barycentric cs:ca=0.0,cb=1.0,cc=0.0) coordinate (cb2) circle (2pt) node[above]             {$(x_2,y_2)$};
          \draw[fill] (barycentric cs:ca=0.0,cb=0.0,cc=1.0) coordinate (cb3) circle (2pt) node[below]             {$(x_3,y_3)$};
          \draw[fill] (barycentric cs:ca=0.5,cb=0.0,cc=0.5) coordinate (cb4) circle (2pt); %node[below]             {$(x_4,y_4)$};
          \draw[fill] (barycentric cs:ca=0.5,cb=0.5,cc=0.0) coordinate (cb5) circle (2pt); %node[above right]       {$(x_5,y_5)$};
          \draw[fill] (barycentric cs:ca=0.0,cb=0.5,cc=0.5) coordinate (cb6) circle (2pt); %node[above left]        {$(x_6,y_6)$};
          \draw[blue,semithick] (cb1) -- (cb4) -- (cbb) -- (cb5) -- cycle;
          \draw[blue,semithick] (cb2) -- (cb6) -- (cbb) -- (cb5) -- cycle;
          \draw[blue,semithick] (cb3) -- (cb4) -- (cbb) -- (cb6) -- cycle;
          %\node at (barycentric cs:cb1=2,cb4=1,cbb=1,cb5=1) {$D_1$};
          %\node at (barycentric cs:cb2=2,cb6=1,cbb=1,cb5=1) {$D_2$};
          %\node at (barycentric cs:cb3=2,cb4=1,cbb=1,cb6=1) {$D_3$};
        \end{scope}
        \begin{scope}[shift={(3.5,5)}] 
            \node at (3,0.5) {Barycentric Space $\mathbb{B}^3$};
            \draw[black,thick] (3,4) coordinate (ca) 
              -- +(-60:3cm) coordinate (cb) 
              -- +(-120:3cm) coordinate (cc) -- cycle;
            %\path (0,0) node[above] {$1$};
            %\path (-60:3cm) node[above right] {$2$};
            %\path (-120:3cm) node[above left] {$3$};
            %\path (-60:1.5cm) node[above right] {$5$};
            %\path (-120:1.5cm) node[above left] {$4$};
            %\path (-60:3cm) -- (-120:3cm) node[midway,above] {$6$};
            \draw[fill] (barycentric cs:ca=0.333,cb=0.333,cc=0.333) coordinate (cbb) circle (2pt) node[below right] {$B_C$};
            \draw[fill] (barycentric cs:ca=1.0,cb=0.0,cc=0.0) coordinate (cb1) circle (2pt) node[above] {$B_1$};
            \draw[fill] (barycentric cs:ca=0.0,cb=1.0,cc=0.0) coordinate (cb2) circle (2pt) node[above right] {$B_2$};
            \draw[fill] (barycentric cs:ca=0.0,cb=0.0,cc=1.0) coordinate (cb3) circle (2pt) node[above left] {$B_3$};
            \draw[fill] (barycentric cs:ca=0.5,cb=0.0,cc=0.5) coordinate (cb4) circle (2pt) node[above left] {$B_4$};
            \draw[fill] (barycentric cs:ca=0.5,cb=0.5,cc=0.0) coordinate (cb5) circle (2pt) node[above right] {$B_5$};
            \draw[fill] (barycentric cs:ca=0.0,cb=0.5,cc=0.5) coordinate (cb6) circle (2pt) node[below] {$B_6$};
            \draw[blue,semithick] (cb1) -- (cb4) -- (cbb) -- (cb5) -- cycle;
            \draw[blue,semithick] (cb2) -- (cb6) -- (cbb) -- (cb5) -- cycle;
            \draw[blue,semithick] (cb3) -- (cb4) -- (cbb) -- (cb6) -- cycle;
            %
            %\draw[fill,red] (barycentric cs:ca=0.419,cb=0.291,cc=0.291) coordinate (cb1) circle (2pt);
            %\draw[fill,red] (barycentric cs:ca=0.556,cb=0.078,cc=0.367) coordinate (cb1) circle (2pt);
            %\draw[fill,red] (barycentric cs:ca=0.556,cb=0.367,cc=0.078) coordinate (cb1) circle (2pt);
            %\draw[fill,red] (barycentric cs:ca=0.804,cb=0.098,cc=0.098) coordinate (cb1) circle (2pt);
            %\draw[fill,red] (barycentric cs:ca=0.291,cb=0.419,cc=0.291) coordinate (cb1) circle (2pt);
            %\draw[fill,red] (barycentric cs:ca=0.367,cb=0.556,cc=0.078) coordinate (cb1) circle (2pt);
            %\draw[fill,red] (barycentric cs:ca=0.078,cb=0.556,cc=0.367) coordinate (cb1) circle (2pt);
            %\draw[fill,red] (barycentric cs:ca=0.098,cb=0.804,cc=0.098) coordinate (cb1) circle (2pt);
            %\draw[fill,red] (barycentric cs:ca=0.291,cb=0.291,cc=0.419) coordinate (cb1) circle (2pt);
            %\draw[fill,red] (barycentric cs:ca=0.078,cb=0.367,cc=0.556) coordinate (cb1) circle (2pt);
            %\draw[fill,red] (barycentric cs:ca=0.367,cb=0.078,cc=0.556) coordinate (cb1) circle (2pt);
            %\draw[fill,red] (barycentric cs:ca=0.098,cb=0.098,cc=0.804) coordinate (cb1) circle (2pt);
        \end{scope}
        \begin{scope}[shift={(10,0)}] 
            \node at (0,-2) {Integration Space ${[-1,1]}^2$};
            \begin{scope}[shift={(-1.75,0)}]
                \draw[-latex] (-1.5,0) -- (1.5,0) node[below] {$X$};
                \draw[-latex] (0,-1.5) -- (0,1.5) node[left]  {$Y$};
                \draw[blue,semithick] (-1,-1) node[below left] {$B_1$}
                  -- (-1,1) node[above left] {$B_5$}
                  -- (1,1) node[above right] {$B_C$}
                  -- (1,-1) node[below right] {$B_4$} -- cycle;
            \end{scope}
            \begin{scope}[shift={(1.75,0)}]
                \draw[-latex] (-1.5,0) -- (1.5,0) node[below] {$X$};
                \draw[-latex] (0,-1.5) -- (0,1.5) node[left]  {$Y$};
                \draw[blue,semithick] (-1,-1) node[below left] {$B_2$}
                  -- (-1,1) node[above left] {$B_6$}
                  -- (1,1) node[above right] {$B_C$}
                  -- (1,-1) node[below right] {$B_5$} -- cycle;
            \end{scope}
            \begin{scope}[shift={(0,3.5)}]
                \draw[-latex] (-1.5,0) -- (1.5,0) node[below] {$X$};
                \draw[-latex] (0,-1.5) -- (0,1.5) node[left]  {$Y$};
                \draw[blue,semithick] (-1,-1) node[below left] {$B_3$}
                  -- (-1,1) node[above left] {$B_4$}
                  -- (1,1)  node[above right] {$B_C$}
                  -- (1,-1) node[below right] {$B_6$} -- cycle;
            \end{scope}
        \end{scope}
        \draw[<-] (4,5) -- (4.5,6) node[midway,above left] {$T_{C1}$};
        \draw[<-] (8.5,6) -- (9,5) node[midway,above right] {$T_{C2}$};
        \draw[<-] (4,0) -- (6,-0.5) node[midway,below] {$T_3$};
    \end{tikzpicture}
\end{Figure}

Let $S$ denote a Matrix of the coordinates of the vertices of the simplex in the following form\begin{equation}
    S = \begin{bmatrix}
        x_1 & x_2 & x_3 \\
        y_1 & y_2 & y_3 \\
    \end{bmatrix}
\end{equation}

The Coordinates $C_{i}$ of all characteristic points in Equations \eqref{eqn:characteristicPoint1},\eqref{eqn:characteristicPoint2} and \eqref{eqn:characteristicPoint3} inside the simplex can be expressed in the following form \begin{align}
    C_{C} &= S \cdot B_{C} \label{eqn:Trans1}\\
    C_{i} &= S \cdot B_{i} \quad \forall i \in 1,2,\ldots,6 \label{eqn:Trans2}
\end{align}.

The transformation from points in the Integration Domain to the points in the barycentric given in \eqref{eqn:DomainTrans1},\eqref{eqn:DomainTrans2} and \eqref{eqn:DomainTrans3} can be expressed as \begin{align}
    B_{D_1}(X,Y) &=\begin{bmatrix}
        B_1 & B_5 & B_4 & B_C \\ 
    \end{bmatrix} \begin{bmatrix}
        G_1(X,Y) \\
        G_2(X,Y) \\
        G_3(X,Y) \\
        G_4(X,Y) \\
    \end{bmatrix} \\
    B_{D_2}(X,Y) &=\begin{bmatrix}
        B_2 & B_6 & B_5 & B_C \\ 
    \end{bmatrix} \begin{bmatrix}
        G_1(X,Y) \\
        G_2(X,Y) \\
        G_3(X,Y) \\
        G_4(X,Y) \\
    \end{bmatrix} \\
    B_{D_3}(X,Y) &=\begin{bmatrix}
        B_3 & B_4 & B_6 & B_C \\ 
    \end{bmatrix} \begin{bmatrix}
        G_1(X,Y) \\
        G_2(X,Y) \\
        G_3(X,Y) \\
        G_4(X,Y) \\
    \end{bmatrix}
\end{align} 

With the Relationship given in \eqref{eqn:Trans1} and \eqref{eqn:Trans2} one can rewrite this as \begin{align}
    C_{D_1}(X,Y) &= S \begin{bmatrix}
        B_1 & B_5 & B_4 & B_C \\ 
    \end{bmatrix} \begin{bmatrix}
        G_1(X,Y) \\
        G_2(X,Y) \\
        G_3(X,Y) \\
        G_4(X,Y) \\
    \end{bmatrix} \\
    C_{D_2}(X,Y) &= S \begin{bmatrix}
        B_2 & B_6 & B_5 & B_C \\ 
    \end{bmatrix} \begin{bmatrix}
        G_1(X,Y) \\
        G_2(X,Y) \\
        G_3(X,Y) \\
        G_4(X,Y) \\
    \end{bmatrix} \\
    C_{D_3}(X,Y) &= S \begin{bmatrix}
        B_3 & B_4 & B_6 & B_C \\ 
    \end{bmatrix} \begin{bmatrix}
        G_1(X,Y) \\
        G_2(X,Y) \\
        G_3(X,Y) \\
        G_4(X,Y) \\
    \end{bmatrix}
\end{align}

With the Jacobi Matrix of any particular Domain being \begin{equation}
    J(C_{D_i}) = \begin{bmatrix}
        \pardiff{C_{D_i}}{X} & \pardiff{C_{D_i}}{Y} \\ 
    \end{bmatrix} = \begin{bmatrix}
        C_i & C_j & C_k & C_C \\ 
    \end{bmatrix}\cdot \begin{bmatrix}
        \pardiff{G_1}{X} & \pardiff{G_1}{Y} \\  
        \pardiff{G_2}{X} & \pardiff{G_2}{Y} \\
        \pardiff{G_3}{X} & \pardiff{G_3}{Y} \\
        \pardiff{G_4}{X} &\pardiff{G_4}{Y} \\
    \end{bmatrix}
\end{equation}

\paragraph{Gauss Point Distribution} for the three Orders of Integration used.

\begin{Figure}
    \centering
    \begin{tikzpicture}
        \begin{scope}[shift={(0,0)}] 
            \node at (3,0.5) {Gauss Points $n=1$};
            \draw[black,thick] (3,4) coordinate (ca) 
              -- +(-60:3cm) coordinate (cb) 
              -- +(-120:3cm) coordinate (cc) -- cycle;
            %\path (0,0) node[above] {$1$};
            %\path (-60:3cm) node[above right] {$2$};
            %\path (-120:3cm) node[above left] {$3$};
            %\path (-60:1.5cm) node[above right] {$5$};
            %\path (-120:1.5cm) node[above left] {$4$};
            %\path (-60:3cm) -- (-120:3cm) node[midway,above] {$6$};
            %\draw[blue,semithick] (cb1) -- (cb4) -- (cbb) -- (cb5) -- cycle;
            %\draw[blue,semithick] (cb2) -- (cb6) -- (cbb) -- (cb5) -- cycle;
            %\draw[blue,semithick] (cb3) -- (cb4) -- (cbb) -- (cb6) -- cycle;
            %
            \draw[fill,red] (barycentric cs:ca=0.583,cb=0.208,cc=0.208) coordinate (cb1) circle (2pt);
            \draw[fill,red] (barycentric cs:ca=0.208,cb=0.583,cc=0.208) coordinate (cb1) circle (2pt);
            \draw[fill,red] (barycentric cs:ca=0.208,cb=0.208,cc=0.583) coordinate (cb1) circle (2pt);
        \end{scope}
        \begin{scope}[shift={(3.5,0)}] 
            \node at (3,0.5) {Gauss Points $n=2$};
            \draw[black,thick] (3,4) coordinate (ca) 
              -- +(-60:3cm) coordinate (cb) 
              -- +(-120:3cm) coordinate (cc) -- cycle;
            %\path (0,0) node[above] {$1$};
            %\path (-60:3cm) node[above right] {$2$};
            %\path (-120:3cm) node[above left] {$3$};
            %\path (-60:1.5cm) node[above right] {$5$};
            %\path (-120:1.5cm) node[above left] {$4$};
            %\path (-60:3cm) -- (-120:3cm) node[midway,above] {$6$};
            %\draw[blue,semithick] (cb1) -- (cb4) -- (cbb) -- (cb5) -- cycle;
            %\draw[blue,semithick] (cb2) -- (cb6) -- (cbb) -- (cb5) -- cycle;
            %\draw[blue,semithick] (cb3) -- (cb4) -- (cbb) -- (cb6) -- cycle;
            %
            \draw[fill,red] (barycentric cs:ca=0.419,cb=0.291,cc=0.291) coordinate (cb1) circle (2pt);
            \draw[fill,red] (barycentric cs:ca=0.556,cb=0.078,cc=0.367) coordinate (cb1) circle (2pt);
            \draw[fill,red] (barycentric cs:ca=0.556,cb=0.367,cc=0.078) coordinate (cb1) circle (2pt);
            \draw[fill,red] (barycentric cs:ca=0.804,cb=0.098,cc=0.098) coordinate (cb1) circle (2pt);
            \draw[fill,red] (barycentric cs:ca=0.291,cb=0.419,cc=0.291) coordinate (cb1) circle (2pt);
            \draw[fill,red] (barycentric cs:ca=0.367,cb=0.556,cc=0.078) coordinate (cb1) circle (2pt);
            \draw[fill,red] (barycentric cs:ca=0.078,cb=0.556,cc=0.367) coordinate (cb1) circle (2pt);
            \draw[fill,red] (barycentric cs:ca=0.098,cb=0.804,cc=0.098) coordinate (cb1) circle (2pt);
            \draw[fill,red] (barycentric cs:ca=0.291,cb=0.291,cc=0.419) coordinate (cb1) circle (2pt);
            \draw[fill,red] (barycentric cs:ca=0.078,cb=0.367,cc=0.556) coordinate (cb1) circle (2pt);
            \draw[fill,red] (barycentric cs:ca=0.367,cb=0.078,cc=0.556) coordinate (cb1) circle (2pt);
            \draw[fill,red] (barycentric cs:ca=0.098,cb=0.098,cc=0.804) coordinate (cb1) circle (2pt);
        \end{scope}
        \begin{scope}[shift={(7,0)}] 
            \node at (3,0.5) {Gauss Points $n=3$};
            \draw[black,thick] (3,4) coordinate (ca) 
              -- +(-60:3cm) coordinate (cb) 
              -- +(-120:3cm) coordinate (cc) -- cycle;
            %\path (0,0) node[above] {$1$};
            %\path (-60:3cm) node[above right] {$2$};
            %\path (-120:3cm) node[above left] {$3$};
            %\path (-60:1.5cm) node[above right] {$5$};
            %\path (-120:1.5cm) node[above left] {$4$};
            %\path (-60:3cm) -- (-120:3cm) node[midway,above] {$6$};
            %\draw[blue,semithick] (cb1) -- (cb4) -- (cbb) -- (cb5) -- cycle;
            %\draw[blue,semithick] (cb2) -- (cb6) -- (cbb) -- (cb5) -- cycle;
            %\draw[blue,semithick] (cb3) -- (cb4) -- (cbb) -- (cb6) -- cycle;
            %
            \draw[fill,red] (barycentric cs:ca=0.375,cb=0.312,cc=0.312) coordinate (cb1) circle (2pt);
            \draw[fill,red] (barycentric cs:ca=0.454,cb=0.176,cc=0.370) coordinate (cb1) circle (2pt);
            \draw[fill,red] (barycentric cs:ca=0.533,cb=0.040,cc=0.427) coordinate (cb1) circle (2pt);
            \draw[fill,red] (barycentric cs:ca=0.454,cb=0.370,cc=0.176) coordinate (cb1) circle (2pt);
            \draw[fill,red] (barycentric cs:ca=0.583,cb=0.208,cc=0.208) coordinate (cb1) circle (2pt);
            \draw[fill,red] (barycentric cs:ca=0.712,cb=0.047,cc=0.241) coordinate (cb1) circle (2pt);
            \draw[fill,red] (barycentric cs:ca=0.533,cb=0.427,cc=0.040) coordinate (cb1) circle (2pt);
            \draw[fill,red] (barycentric cs:ca=0.712,cb=0.241,cc=0.047) coordinate (cb1) circle (2pt);
            \draw[fill,red] (barycentric cs:ca=0.892,cb=0.054,cc=0.054) coordinate (cb1) circle (2pt);
            \draw[fill,red] (barycentric cs:ca=0.312,cb=0.375,cc=0.312) coordinate (cb1) circle (2pt);
            \draw[fill,red] (barycentric cs:ca=0.370,cb=0.454,cc=0.176) coordinate (cb1) circle (2pt);
            \draw[fill,red] (barycentric cs:ca=0.427,cb=0.533,cc=0.040) coordinate (cb1) circle (2pt);
            \draw[fill,red] (barycentric cs:ca=0.176,cb=0.454,cc=0.370) coordinate (cb1) circle (2pt);
            \draw[fill,red] (barycentric cs:ca=0.208,cb=0.583,cc=0.208) coordinate (cb1) circle (2pt);
            \draw[fill,red] (barycentric cs:ca=0.241,cb=0.712,cc=0.047) coordinate (cb1) circle (2pt);
            \draw[fill,red] (barycentric cs:ca=0.040,cb=0.533,cc=0.427) coordinate (cb1) circle (2pt);
            \draw[fill,red] (barycentric cs:ca=0.047,cb=0.712,cc=0.241) coordinate (cb1) circle (2pt);
            \draw[fill,red] (barycentric cs:ca=0.054,cb=0.892,cc=0.054) coordinate (cb1) circle (2pt);
            \draw[fill,red] (barycentric cs:ca=0.312,cb=0.312,cc=0.375) coordinate (cb1) circle (2pt);
            \draw[fill,red] (barycentric cs:ca=0.176,cb=0.370,cc=0.454) coordinate (cb1) circle (2pt);
            \draw[fill,red] (barycentric cs:ca=0.040,cb=0.427,cc=0.533) coordinate (cb1) circle (2pt);
            \draw[fill,red] (barycentric cs:ca=0.370,cb=0.176,cc=0.454) coordinate (cb1) circle (2pt);
            \draw[fill,red] (barycentric cs:ca=0.208,cb=0.208,cc=0.583) coordinate (cb1) circle (2pt);
            \draw[fill,red] (barycentric cs:ca=0.047,cb=0.241,cc=0.712) coordinate (cb1) circle (2pt);
            \draw[fill,red] (barycentric cs:ca=0.427,cb=0.040,cc=0.533) coordinate (cb1) circle (2pt);
            \draw[fill,red] (barycentric cs:ca=0.241,cb=0.047,cc=0.712) coordinate (cb1) circle (2pt);
            \draw[fill,red] (barycentric cs:ca=0.054,cb=0.054,cc=0.892) coordinate (cb1) circle (2pt);
        \end{scope}
    \end{tikzpicture}
\end{Figure}

\subsubsection{Sphere Integrator}

\subsubsection{Other Quadrature Formulas}

\pagebreak

\subsection{Subdivision Integration Strategy}
\label{sec:subdivision2D}
Because any triangle can be decomposed into 4 similar triangles, the subdivision algorithm turns out to be quite practical in implementation.

The Integral over a parent Simplex $\boldsymbol{\Delta}_p$, can be expressed as an Integral over 4 Child Simpleces $\boldsymbol{\Delta}_i$: 
\begin{marginfigure}
    \centering
    \begin{tikzpicture}
        \draw[black,thick] (0,0) coordinate (ca) 
          -- (-60:3cm) coordinate (cb) 
          -- (-120:3cm) coordinate (cc) -- cycle;
        %\path (0,0) node[above] {$1$};
        %\path (-60:3cm) node[above right] {$2$};
        %\path (-120:3cm) node[above left] {$3$};
        %\path (-60:1.5cm) node[above right] {$5$};
        %\path (-120:1.5cm) node[above left] {$4$};
        %\path (-60:3cm) -- (-120:3cm) node[midway,above] {$6$};
        %\draw[fill] (barycentric cs:ca=0.333,cb=0.333,cc=0.333) coordinate (cbb) circle (2pt) node[below right] {$B_C$};
        \draw[fill] (barycentric cs:ca=1.0,cb=0.0,cc=0.0) coordinate (cb1) circle (2pt) node[above] {$B_1$};
        \draw[fill] (barycentric cs:ca=0.0,cb=1.0,cc=0.0) coordinate (cb2) circle (2pt) node[above right] {$B_2$};
        \draw[fill] (barycentric cs:ca=0.0,cb=0.0,cc=1.0) coordinate (cb3) circle (2pt) node[above left] {$B_3$};
        \draw[fill] (barycentric cs:ca=0.5,cb=0.0,cc=0.5) coordinate (cb4) circle (2pt) node[above left] {$B_4$};
        \draw[fill] (barycentric cs:ca=0.5,cb=0.5,cc=0.0) coordinate (cb5) circle (2pt) node[above right] {$B_5$};
        \draw[fill] (barycentric cs:ca=0.0,cb=0.5,cc=0.5) coordinate (cb6) circle (2pt) node[below] {$B_6$};
        \draw[blue,semithick] (cb1) -- (cb5) -- (cb4) -- cycle;
        \draw[blue,semithick] (cb2) -- (cb6) -- (cb5) -- cycle;
        \draw[blue,semithick] (cb3) -- (cb4) -- (cb6) -- cycle;
        \draw[blue,semithick] (cb4) -- (cb5) -- (cb6) -- cycle;
        \node at (barycentric cs:cb1=1,cb5=1,cb4=1) {$\boldsymbol{\Delta}_1$};
        \node at (barycentric cs:cb2=1,cb6=1,cb5=1) {$\boldsymbol{\Delta}_2$};
        \node at (barycentric cs:cb3=1,cb4=1,cb6=1) {$\boldsymbol{\Delta}_3$};
        \node at (barycentric cs:cb4=1,cb5=1,cb6=1) {$\boldsymbol{\Delta}_4$};
    \end{tikzpicture}
\end{marginfigure}
\begin{align}
    \iint_{\boldsymbol{\Delta}_p} F \diff\boldsymbol{\Delta}_p &= \iint_{\boldsymbol{\Delta}_1} F \diff\boldsymbol{\Delta}_1 + \iint_{\boldsymbol{\Delta}_2} F \diff\boldsymbol{\Delta}_2 +\iint_{\boldsymbol{\Delta}_3} F \diff\boldsymbol{\Delta}_3 + \iint_{\boldsymbol{\Delta}_4} F \diff\boldsymbol{\Delta}_4
\end{align}

The Coordinates of a Child Simplex $\simplex_i$ can be expressed in local-barycentric coordinates $\xi_{i,1}',\, \xi_{i,2}',\, \xi_{i,3}'$

The corresponding Transformation from the local coordinate System into the global is given by \begin{align}
   T_{lg}(\xi_{i,1}', \xi_{i,2}', \xi_{i,3}') &= B_{i,1}\xi_{i,1}' + B_{i,2}\xi_{i,2}' + B_{i,3} \xi_{i,3}'
\end{align} where $B_{i,j}$ are the coordinates of the Verteces of the Child Simplex $\simplex_i$.

This can be done recursively, to get a desired accuracy.
%
%The Vertices of e.g. the $[B_3,B_4,B_6]$ Child Simplex can be calculated by \begin{align}
%    \begin{bmatrix}
%        B_1' & B_2' & B_3' \\ 
%    \end{bmatrix} &= \begin{bmatrix}
%        B_3 & B_4 & B_6 \\ 
%    \end{bmatrix} \begin{bmatrix}
%        B_1 & B_2 & B_3 \\ 
%    \end{bmatrix}
%\end{align}



\subsection{Simplex Subdivision}

A Criterion for adaptive integration from \cite{gonnetReviewErrorEstimation2012}\begin{align}
    Q &= \iint_{\simplex} F \diff \simplex \\
    \varepsilon &= \Bigl\vert Q - \iint_{\simplex_p} F \diff \simplex_p \Bigr\vert
\end{align}

The Coordinates of the Verteces of the 4 Child Simplizes $\simplex_i$ can be calculated with $S$ by \begin{align}
    S_{\simplex,1} &= S \cdot B_{\simplex,1} = S \cdot \begin{bmatrix}
        B_1 & B_5 & B_4 \\
    \end{bmatrix} = S \cdot \begin{bmatrix}
        1 & 0.5 & 0.5 \\
        0 & 0.5 & 0 \\
        0 & 0 & 0.5 \\
    \end{bmatrix} \\
    S_{\simplex,2} &= S \cdot B_{\simplex,2} = S \cdot \begin{bmatrix}
        B_2 & B_6 & B_5 \\
    \end{bmatrix} = S \cdot \begin{bmatrix}
        0 & 0   & 0.5 \\
        1 & 0.5 & 0.5 \\
        0 & 0.5 & 0   \\
    \end{bmatrix} \\
    S_{\simplex,3} &= S \cdot B_{\simplex,3} = S \cdot \begin{bmatrix}
        B_3 & B_4 & B_6 \\
    \end{bmatrix} = S \cdot \begin{bmatrix}
        0 & 0.5 & 0   \\
        0 & 0   & 0.5 \\
        1 & 0.5 & 0.5 \\
    \end{bmatrix} \\
    S_{\simplex,4} &= S \cdot B_{\simplex,4} = S \cdot \begin{bmatrix}
        B_4 & B_5 & B_6 \\
    \end{bmatrix} = S \cdot \begin{bmatrix}
        0.5 & 0.5 & 0   \\
        0   & 0.5 & 0.5 \\
        0.5 & 0   & 0.5 \\
    \end{bmatrix}
\end{align}

Coordinates of any Grandchild-Simplizes can be calculated by chaining $B_{\simplex,i}$ Transformations

\begin{Figure}
    \centering
    \begin{tikzpicture}
        \draw[black,thick] (3,4) coordinate (ca) 
              -- +(-60:3cm) coordinate (cb) 
              -- +(-120:3cm) coordinate (cc) -- cycle;
              %\draw[fill,blue] (barycentric cs:ca=0.583,cb=0.208,cc=0.208) coordinate (cb1) circle (2pt);
              %\draw[fill,blue] (barycentric cs:ca=0.208,cb=0.583,cc=0.208) coordinate (cb1) circle (2pt);
              %\draw[fill,blue] (barycentric cs:ca=0.208,cb=0.208,cc=0.583) coordinate (cb1) circle (2pt);
              \draw[fill,red] (barycentric cs:ca=0.792,cb=0.104,cc=0.104) coordinate (cb1) circle (2pt);
              \draw[fill,red] (barycentric cs:ca=0.604,cb=0.292,cc=0.104) coordinate (cb1) circle (2pt);
              \draw[fill,red] (barycentric cs:ca=0.604,cb=0.104,cc=0.292) coordinate (cb1) circle (2pt);
              \draw[fill,red] (barycentric cs:ca=0.104,cb=0.792,cc=0.104) coordinate (cb1) circle (2pt);
              \draw[fill,red] (barycentric cs:ca=0.104,cb=0.604,cc=0.292) coordinate (cb1) circle (2pt);
              \draw[fill,red] (barycentric cs:ca=0.292,cb=0.604,cc=0.104) coordinate (cb1) circle (2pt);
              \draw[fill,red] (barycentric cs:ca=0.104,cb=0.104,cc=0.792) coordinate (cb1) circle (2pt);
              \draw[fill,red] (barycentric cs:ca=0.292,cb=0.104,cc=0.604) coordinate (cb1) circle (2pt);
              \draw[fill,red] (barycentric cs:ca=0.104,cb=0.292,cc=0.604) coordinate (cb1) circle (2pt);
              \draw[fill,red] (barycentric cs:ca=0.396,cb=0.208,cc=0.396) coordinate (cb1) circle (2pt);
              \draw[fill,red] (barycentric cs:ca=0.396,cb=0.396,cc=0.208) coordinate (cb1) circle (2pt);
              \draw[fill,red] (barycentric cs:ca=0.208,cb=0.396,cc=0.396) coordinate (cb1) circle (2pt);
    \end{tikzpicture}
\end{Figure}

\pagebreak

\section{Simplex Integration in $n=3$}

\begin{Figure}
    \centering
    \begin{tikzpicture}
        \begin{scope}[engineering axonometry]
            \begin{scope}[scale=2]
                \coordinate (b1) at ({sqrt(8/9)},0,0);
                \coordinate (b2) at ({-sqrt(2/9)},{sqrt(2/3)},0);
                \coordinate (b3) at ({-sqrt(2/9)},{-sqrt(2/3)},0);
                \coordinate (b4) at (0,0,4/3);
            \end{scope}
            \draw (b1) -- (b2);
            \draw (b2) -- (b3);
            \draw (b3) -- (b1);
            \draw (b1) -- (b4);
            \draw (b2) -- (b4);
            \draw (b3) -- (b4);
            \node[below] at (b1) {$B_1$};
            \node[right] at (b2) {$B_2$};
            \node[left]  at (b3) {$B_3$};
            \node[above] at (b4) {$B_4$};
            %% Centroid
            \draw[fill] (barycentric cs:b1=0.25,b2=0.25,b3=0.25,b4=0.25) coordinate (bc) circle (2pt);% node[below right] {$B_C$};
            %% Center of Planes
            \draw[fill] (barycentric cs:b1=0.33,b2=0.33,b3=0.33,b4=0.00) coordinate (ba4) circle (2pt);% node[below right] {$B_1$};
            \draw[fill] (barycentric cs:b1=0.00,b2=0.33,b3=0.33,b4=0.33) coordinate (ba1) circle (2pt);% node[below right] {$B_2$};
            \draw[fill] (barycentric cs:b1=0.33,b2=0.00,b3=0.33,b4=0.33) coordinate (ba2) circle (2pt);% node[below right] {$B_3$};
            \draw[fill] (barycentric cs:b1=0.33,b2=0.33,b3=0.00,b4=0.33) coordinate (ba3) circle (2pt);% node[below right] {$B_4$};
            %% Center of Edges
            \draw[fill] (barycentric cs:b1=0.50,b2=0.50,b3=0.00,b4=0.00) coordinate (be12) circle (2pt);% node[below right] {$B_1$};
            \draw[fill] (barycentric cs:b1=0.00,b2=0.50,b3=0.50,b4=0.00) coordinate (be23) circle (2pt);% node[below right] {$B_2$};
            \draw[fill] (barycentric cs:b1=0.00,b2=0.00,b3=0.50,b4=0.50) coordinate (be34) circle (2pt);% node[below right] {$B_3$};
            \draw[fill] (barycentric cs:b1=0.50,b2=0.00,b3=0.50,b4=0.00) coordinate (be13) circle (2pt);
            \draw[fill] (barycentric cs:b1=0.50,b2=0.00,b3=0.00,b4=0.50) coordinate (be14) circle (2pt);% node[below right] {$B_3$};
            \draw[fill] (barycentric cs:b1=0.00,b2=0.50,b3=0.00,b4=0.50) coordinate (be24) circle (2pt);
            %% Domains
            \draw[semithick,blue,line join=round] (b1) -- (be12) -- (ba3) -- (bc) -- (ba2) -- (be13) --cycle;
            \draw[semithick,blue,line join=round] (be12) -- (ba4) -- (be13);
            \draw[semithick,blue,line join=round] (ba3) -- (be14) -- (ba2);
            \draw[semithick,blue,line join=round] (b1) -- (be14);
            \draw[semithick,blue,line join=round] (ba4) -- (bc);
            %% Domains
            \draw[semithick,blue,line join=round] (b2) -- (be12) -- (ba3) -- (bc) -- (ba1) -- (be23) --cycle;
            \draw[semithick,blue,line join=round] (be12) -- (ba4) -- (be13);
            \draw[semithick,blue,line join=round] (b2) -- (be24) -- (ba3);
            \draw[semithick,blue,line join=round] (be23) -- (ba4);
            \draw[semithick,blue,line join=round] (ba4) -- (bc);
            %% Domains
            \draw[semithick,blue,line join=round] (ba2) -- (be34) -- (ba1) -- (be24) -- (b4) -- (be14) --cycle;
            %%
            \draw[semithick,blue,line join=round] (b4) -- (be34) -- (b3) -- (be13);
            \draw[semithick,blue,line join=round] (b3) -- (be23);
        \end{scope}
        \begin{scope}[shift={(5.0,0.0)},engineering axonometry]
            \begin{scope}[scale=2]
                \coordinate (b1) at ({sqrt(8/9)},0,0);
                \coordinate (b2) at ({-sqrt(2/9)},{sqrt(2/3)},0);
                \coordinate (b3) at ({-sqrt(2/9)},{-sqrt(2/3)},0);
                \coordinate (b4) at (0,0,4/3);
            \end{scope}
            \draw (b1) -- (b2);
            \draw (b2) -- (b3);
            \draw (b3) -- (b1);
            \draw (b1) -- (b4);
            \draw (b2) -- (b4);
            \draw (b3) -- (b4);
            \node[below] at (b1) {$B_1$};
            \node[right] at (b2) {$B_2$};
            \node[left]  at (b3) {$B_3$};
            \node[above] at (b4) {$B_4$};
            %% Centroid
            %\draw[fill] (barycentric cs:b1=0.25,b2=0.25,b3=0.25,b4=0.25) coordinate (bc) circle (2pt);% node[below right] {$B_C$};
            %% Center of Planes
            %\draw[fill] (barycentric cs:b1=0.33,b2=0.33,b3=0.33,b4=0.00) coordinate (ba4) circle (2pt);% node[below right] {$B_1$};
            %\draw[fill] (barycentric cs:b1=0.00,b2=0.33,b3=0.33,b4=0.33) coordinate (ba1) circle (2pt);% node[below right] {$B_2$};
            %\draw[fill] (barycentric cs:b1=0.33,b2=0.00,b3=0.33,b4=0.33) coordinate (ba2) circle (2pt);% node[below right] {$B_3$};
            %\draw[fill] (barycentric cs:b1=0.33,b2=0.33,b3=0.00,b4=0.33) coordinate (ba3) circle (2pt);% node[below right] {$B_4$};
            %% Center of Edges
            \draw[fill] (barycentric cs:b1=0.50,b2=0.50,b3=0.00,b4=0.00) coordinate (be12) circle (2pt);% node[below right] {$B_1$};
            \draw[fill] (barycentric cs:b1=0.00,b2=0.50,b3=0.50,b4=0.00) coordinate (be23) circle (2pt);% node[below right] {$B_2$};
            \draw[fill] (barycentric cs:b1=0.00,b2=0.00,b3=0.50,b4=0.50) coordinate (be34) circle (2pt);% node[below right] {$B_3$};
            \draw[fill] (barycentric cs:b1=0.50,b2=0.00,b3=0.50,b4=0.00) coordinate (be13) circle (2pt);
            \draw[fill] (barycentric cs:b1=0.50,b2=0.00,b3=0.00,b4=0.50) coordinate (be14) circle (2pt);% node[below right] {$B_3$};
            \draw[fill] (barycentric cs:b1=0.00,b2=0.50,b3=0.00,b4=0.50) coordinate (be24) circle (2pt);
            %% Domains
            \newcommand{\drawsimplex}[4]{
                \draw[semithick,blue,line join=round] (#1) -- (#2);
                \draw[semithick,blue,line join=round] (#1) -- (#3);
                \draw[semithick,blue,line join=round] (#1) -- (#4);
                \draw[semithick,blue,line join=round] (#2) -- (#3);
                \draw[semithick,blue,line join=round] (#2) -- (#4);
                \draw[semithick,blue,line join=round] (#3) -- (#4);
            }
            \drawsimplex{b1}{be13}{be12}{be14}
            \drawsimplex{b2}{be12}{be23}{be24}
            \drawsimplex{b3}{be13}{be23}{be34}
            \drawsimplex{b4}{be24}{be34}{be14}
            \draw[semithick,blue,line join=round] (be13) -- (be24);
        \end{scope}
    \end{tikzpicture}
\end{Figure}

\subsection{Space Definition}

In a $\R^3$ Simplex, the barycentric coordinates $\Bary^4$ need to be used: \begin{align}
    \Bary^4 &= \{ \xi_1,\,\xi_2,\,\xi_3,\,\xi_4 \in [0,1] | \xi_1 + \xi_2 + \xi_3 + \xi_4 = 1 \}
\end{align} where each set of coordinates corresponds to a point inside the simplex, spanned by the Coordinates in $\R^3$
\begin{align}
    C_1 &= \begin{bmatrix}
        x_1 \\
        y_1 \\
        z_1 \\
    \end{bmatrix} & C_2 &= \begin{bmatrix}
        x_2 \\
        y_2 \\
        z_2 \\
    \end{bmatrix} & C_3 &= \begin{bmatrix}
        x_3 \\
        y_3 \\
        z_3 \\
    \end{bmatrix} & C_4 &= \begin{bmatrix}
        x_4 \\
        y_4 \\
        z_4 \\
    \end{bmatrix}
\end{align}
The points $C_i$ can be written in Matrix Form \begin{align}
    \MatI{C} &= \begin{bmatrix}
        C_1 & C_2 & C_3 & C_4 \\
    \end{bmatrix} = \begin{bmatrix}
        x_1 & x_2 & x_3 & x_4 \\
        y_1 & y_2 & y_3 & y_4 \\
        z_1 & z_2 & z_3 & z_4 \\
    \end{bmatrix}
\end{align}

A mapping $M_B: \Bary^4 \abbild \R^3$ can be written as \begin{align}
    C &= \MatI{C} \cdot \begin{bmatrix}
        \xi_1 \\
        \xi_2 \\
        \xi_3 \\
        \xi_4 \\
    \end{bmatrix}
\end{align}

\subsection{Common Mappings}

The Reference Element in Finite Element Analysis is often given in a $\xi,\, \eta,\, \zeta$ Coordinates. The set of coordinates
spanning this \textbf{Reference Space} $\R^3_{r}$ can be mapped via $M_R: \R^3_{r} \abbild \Bary^4$ to barycentric coordinates \begin{align}
    B &= \begin{bmatrix}
        -1 & -1 & -1 & 1\\
        1 & 0 & 0 & 0 \\
        0 & 1 & 0 & 0 \\
        0 & 0 & 1 & 0 \\
    \end{bmatrix} \cdot \begin{bmatrix}
        \xi \\
        \eta \\
        \zeta \\
        1 \\
    \end{bmatrix} & R &= \begin{bmatrix}
        0 & 1 & 0 & 0 \\
        0 & 0 & 1 & 0 \\
        0 & 0 & 0 & 1 \\
    \end{bmatrix} \cdot \begin{bmatrix}
        \xi_1 \\
        \xi_2 \\
        \xi_3 \\
        \xi_4 \\
    \end{bmatrix}
\end{align}

A Transformation of the subspace $\Bary^4_{S}$-Space  with its supspace barycentric coordinates $\xi_{S,i}$ into its greater space $\Bary^4$ can be denoted by \begin{align}
    B &= \begin{bmatrix}
        B_1 & B_2 & B_3 & B_4
    \end{bmatrix} \cdot \begin{bmatrix}
        \xi_{S,1} \\
        \xi_{S,2} \\
        \xi_{S,3} \\
        \xi_{S,4} \\   
    \end{bmatrix}
\end{align}

\subsection{Pure Integration Strategy}

\subsubsection{Quadrilateral Integration}

Similar to the 2D Case, Quadrilaterals can be identified inside the tetrahedron for which, normal gaussian quadrature rules can be applied.
In this Integration Space $\R^3_{I}\coloneq [-1,1]\times[-1,1]\times[-1,1]$ a transformation must be formulated.

For brevity, the interpolation functions $g_1,\, g_2$ will again be defined \begin{align}
    g_1(a) &= \frac{1}{2} + \frac{a}{2} & g_2(a) &= \frac{1}{2} - \frac{a}{2} 
\end{align}

For which, the Interpolation functions for the points can be defined: \begin{align}
    G_1(X,Y,Z) &= g_2(X)g_2(Y)g_2(Z) & G_2(X,Y,Z) &= g_1(X)g_2(Y)g_2(Z) \\
    G_3(X,Y,Z) &= g_1(X)g_1(Y)g_2(Z) & G_4(X,Y,Z) &= g_2(X)g_1(Y)g_2(Z) \\
    G_5(X,Y,Z) &= g_2(X)g_2(Y)g_1(Z) & G_6(X,Y,Z) &= g_1(X)g_2(Y)g_1(Z) \\
    G_7(X,Y,Z) &= g_1(X)g_1(Y)g_1(Z) & G_8(X,Y,Z) &= g_2(X)g_1(Y)g_1(Z)
\end{align}

\begin{Figure}
    \centering
    \begin{tikzpicture}
        \begin{scope}[engineering axonometry]
            \draw[fill] (0,0,0) +(0,0,-1.4) node {Quadrilateral Domains};
            \coordinate (shift) at (0,0,0.3);
            \begin{scope}[shift={(shift)},scale=3.0]
                \coordinate (b1) at ({sqrt(8/9)},0,0);
                \coordinate (b2) at ({-sqrt(2/9)},{sqrt(2/3)},0);
                \coordinate (b3) at ({-sqrt(2/9)},{-sqrt(2/3)},0);
                \coordinate (b4) at (0,0,4/3);
            \end{scope}
            \draw (b1) -- (b2);
            \draw (b2) -- (b3);
            \draw (b3) -- (b1);
            \draw (b1) -- (b4);
            \draw (b2) -- (b4);
            \draw (b3) -- (b4);
            \node[below] at (b1) {$B_1$};
            \node[right] at (b2) {$B_2$};
            \node[left]  at (b3) {$B_3$};
            \node[above] at (b4) {$B_4$};
            %% Centroid
            \draw[fill] (barycentric cs:b1=0.25,b2=0.25,b3=0.25,b4=0.25) coordinate (b15) circle (2pt) node[above] {$B_{15}$};
            %% Center of Planes
            \draw[fill] (barycentric cs:b1=0.00,b2=0.33,b3=0.33,b4=0.33) coordinate (b11) circle (2pt) node[above] {$B_{11}$};
            \draw[fill] (barycentric cs:b1=0.33,b2=0.00,b3=0.33,b4=0.33) coordinate (b12) circle (2pt) node[below] {$B_{12}$};
            \draw[fill] (barycentric cs:b1=0.33,b2=0.33,b3=0.00,b4=0.33) coordinate (b13) circle (2pt) node[right] {$B_{13}$};
            \draw[fill] (barycentric cs:b1=0.33,b2=0.33,b3=0.33,b4=0.00) coordinate (b14) circle (2pt) node[below] {$B_{14}$};
            %% Center of Edges
            \draw[fill] (barycentric cs:b1=0.50,b2=0.50,b3=0.00,b4=0.00) coordinate (be5) circle (2pt) node[below] {$B_{5}$};
            \draw[fill] (barycentric cs:b1=0.00,b2=0.50,b3=0.50,b4=0.00) coordinate (be6) circle (2pt) node[right=1em] {$B_{6}$};
            \draw[fill] (barycentric cs:b1=0.50,b2=0.00,b3=0.50,b4=0.00) coordinate (be7) circle (2pt) node[below left] {$B_{7}$};
            \draw[fill] (barycentric cs:b1=0.50,b2=0.00,b3=0.00,b4=0.50) coordinate (be8) circle (2pt) node[left] {$B_{8}$};
            \draw[fill] (barycentric cs:b1=0.00,b2=0.50,b3=0.00,b4=0.50) coordinate (be9) circle (2pt) node[above right] {$B_{9}$};
            \draw[fill] (barycentric cs:b1=0.00,b2=0.00,b3=0.50,b4=0.50) coordinate (be10) circle (2pt) node[above left] {$B_{10}$};
            \newcommand{\quadrilat}[8]{
                \draw[blue] (#1) -- (#2) -- (#3) -- (#4) -- cycle;
                \draw[blue] (#5) -- (#6) -- (#7) -- (#8) -- cycle;
                \draw[blue] (#1) -- (#5);
                \draw[blue] (#2) -- (#6);
                \draw[blue] (#3) -- (#7);
                \draw[blue] (#4) -- (#8);
            }
            \quadrilat{b1}{be5}{b14}{be7}{be8}{b13}{b15}{b12}
            \quadrilat{b2}{be6}{b14}{be5}{be9}{b11}{b15}{b13}
            \quadrilat{b3}{be7}{b14}{be6}{be10}{b12}{b15}{b11}
            \quadrilat{b4}{be8}{b12}{be10}{be9}{b13}{b15}{b11}
        \end{scope}
        \begin{scope}[shift={(5.5,0)},engineering axonometry]
            \draw[fill] (0,0,0) +(0,0,-1.4) node {Reference Quadrilateral};
            \coordinate (shift) at (0,0,1);
            \begin{scope}[shift={(shift)},scale=1.2]
                \coordinate (q1) at (-1,-1,-1);
                \coordinate (q2) at (1,-1 ,-1);
                \coordinate (q3) at (1,1  ,-1);
                \coordinate (q4) at (-1,1 ,-1);
                \coordinate (q5) at (-1,-1,1);
                \coordinate (q6) at (1,-1 ,1);
                \coordinate (q7) at (1,1  ,1);
                \coordinate (q8) at (-1,1 ,1);
                \coordinate (q0) at (0,0,0);
            \end{scope}
            \draw[->] (q0) -- +(1,0,0) node[above] {$X$};
            \draw[->] (q0) -- +(0,1,0) node[below] {$Y$};
            \draw[->] (q0) -- +(0,0,1) node[above] {$Z$};
            \draw[blue] (q1) -- (q2) -- (q3) -- (q4) -- cycle;
            \draw[blue] (q5) -- (q6) -- (q7) -- (q8) -- cycle;
            \draw[blue] (q1) -- (q5);
            \draw[blue] (q2) -- (q6);
            \draw[blue] (q3) -- (q7);
            \draw[blue] (q4) -- (q8);
            \draw[fill] (q1) circle (2pt) node[left] {$1$};
            \draw[fill] (q2) circle (2pt) node[left] {$2$};
            \draw[fill] (q3) circle (2pt) node[right] {$3$};
            \draw[fill] (q4) circle (2pt) node[right] {$4$};
            \draw[fill] (q5) circle (2pt) node[left] {$5$};
            \draw[fill] (q6) circle (2pt) node[left] {$6$};
            \draw[fill] (q7) circle (2pt) node[right] {$7$};
            \draw[fill] (q8) circle (2pt) node[right] {$8$};
            %% Centroid
            %\draw[fill] (barycentric cs:b1=0.25,b2=0.25,b3=0.25,b4=0.25) coordinate (bc) circle (2pt) node[right] {$B_{15}$};
            %% Center of Planes
        \end{scope}
    \end{tikzpicture}
\end{Figure}

\begin{equation}
    \iiint_{\simplex} F \diff \simplex = \iiint_{D_1} F \diff D_1 + \iiint_{D_2} F \diff D_2 + \iiint_{D_3} F \diff D_3 + \iiint_{D_4} F \diff D_4
\end{equation} with \begin{align}
    B_{D_1}(X,Y,Z) &= \begin{bmatrix}
        B_1 & B_5 & B_{14} & B_7 & B_8 & B_{13} & B_{15} & B_{12} \\
    \end{bmatrix} \cdot G \\
    B_{D_2}(X,Y,Z) &= \begin{bmatrix}
        B_2 & B_6 & B_{14} & B_5 & B_9 & B_{11} & B_{15} & B_{13} \\
    \end{bmatrix} \cdot G \\
    B_{D_3}(X,Y,Z) &= \begin{bmatrix}
        B_3 & B_7 & B_{14} & B_6 & B_{10} & B_{12} & B_{15} & B_{11} \\
    \end{bmatrix} \cdot G \\
    B_{D_4}(X,Y,Z) &= \begin{bmatrix}
        B_4 & B_8 & B_{12} & B_{10} & B_9 & B_{13} & B_{15} & B_{11} \\
    \end{bmatrix} \cdot G
\end{align} and \begin{equation}
    G(X,Y,Z) = \begin{bmatrix}
        G_1 & G_2 & G_3 & G_4 & G_5 & G_6 & G_7 & G_8 \\
    \end{bmatrix}^T
\end{equation}

The Jacobi-Determinant of any particular Domain being \begin{align}
    J(B_{D_i}) &= B_i \begin{bmatrix}
        \pardiff{G_1}{X} & \pardiff{G_2}{X} & \pardiff{G_3}{X} & \pardiff{G_4}{X} & \pardiff{G_5}{X} & \pardiff{G_6}{X} & \pardiff{G_7}{X} & \pardiff{G_8}{X} \\[1ex]
        \pardiff{G_1}{Y} & \pardiff{G_2}{Y} & \pardiff{G_3}{Y} & \pardiff{G_4}{Y} & \pardiff{G_5}{Y} & \pardiff{G_6}{Y} & \pardiff{G_7}{Y} & \pardiff{G_8}{Y} \\[1ex]
        \pardiff{G_1}{Z} & \pardiff{G_2}{Z} & \pardiff{G_3}{Z} & \pardiff{G_4}{Z} & \pardiff{G_5}{Z} & \pardiff{G_6}{Z} & \pardiff{G_7}{Z} & \pardiff{G_8}{Z} \\
    \end{bmatrix}^T
\end{align}

With the added benefit, that the integration points and their weights can be precomputed.

\begin{Figure}
    \centering
    \begin{tikzpicture}
        \begin{scope}[engineering axonometry]
            \draw[fill] (0,0,0) +(0,0,-1.4) node {Gauss Points $n=1$};
            \coordinate (shift) at (0,0,0.3);
            \begin{scope}[shift={(shift)},scale=3.0]
                \coordinate (b1) at ({sqrt(8/9)},0,0);
                \coordinate (b2) at ({-sqrt(2/9)},{sqrt(2/3)},0);
                \coordinate (b3) at ({-sqrt(2/9)},{-sqrt(2/3)},0);
                \coordinate (b4) at (0,0,4/3);
            \end{scope}
            \draw (b1) -- (b2);
            \draw (b2) -- (b3);
            \draw (b3) -- (b1);
            \draw (b1) -- (b4);
            \draw (b2) -- (b4);
            \draw (b3) -- (b4);
            \draw[fill,red] (barycentric cs:b1=0.469,b2=0.177,b3=0.177,b4=0.177) circle (2pt);
            \draw[fill,red] (barycentric cs:b1=0.177,b2=0.469,b3=0.177,b4=0.177) circle (2pt);
            \draw[fill,red] (barycentric cs:b1=0.177,b2=0.177,b3=0.469,b4=0.177) circle (2pt);
            \draw[fill,red] (barycentric cs:b1=0.177,b2=0.177,b3=0.177,b4=0.469) circle (2pt);
        \end{scope}
        \begin{scope}[shift={(6.0,0)},engineering axonometry]
            \draw[fill] (0,0,0) +(0,0,-1.4) node {Gauss Points $n=2$};
            \coordinate (shift) at (0,0,0.3);
            \begin{scope}[shift={(shift)},scale=3.0]
                \coordinate (b1) at ({sqrt(8/9)},0,0);
                \coordinate (b2) at ({-sqrt(2/9)},{sqrt(2/3)},0);
                \coordinate (b3) at ({-sqrt(2/9)},{-sqrt(2/3)},0);
                \coordinate (b4) at (0,0,4/3);
            \end{scope}
            \draw (b1) -- (b2);
            \draw (b2) -- (b3);
            \draw (b3) -- (b1);
            \draw (b1) -- (b4);
            \draw (b2) -- (b4);
            \draw (b3) -- (b4);
            \draw[fill,blueishGreen] (barycentric cs:b1=0.092,b2=0.725,b3=0.092,b4=0.092) circle (2pt);
            \draw[fill,blueishGreen] (barycentric cs:b1=0.073,b2=0.512,b3=0.073,b4=0.342) circle (2pt);
            \draw[fill,blueishGreen] (barycentric cs:b1=0.342,b2=0.512,b3=0.073,b4=0.073) circle (2pt);
            \draw[fill,blueishGreen] (barycentric cs:b1=0.274,b2=0.391,b3=0.061,b4=0.274) circle (2pt);
            \draw[fill,blueishGreen] (barycentric cs:b1=0.073,b2=0.512,b3=0.342,b4=0.073) circle (2pt);
            \draw[fill,blueishGreen] (barycentric cs:b1=0.061,b2=0.391,b3=0.274,b4=0.274) circle (2pt);
            \draw[fill,blueishGreen] (barycentric cs:b1=0.274,b2=0.391,b3=0.274,b4=0.061) circle (2pt);
            \draw[fill,blueishGreen] (barycentric cs:b1=0.228,b2=0.316,b3=0.228,b4=0.228) circle (2pt);
            \draw[fill,reddishPurple] (barycentric cs:b1=0.092,b2=0.092,b3=0.725,b4=0.092) circle (2pt);
            \draw[fill,reddishPurple] (barycentric cs:b1=0.073,b2=0.073,b3=0.512,b4=0.342) circle (2pt);
            \draw[fill,reddishPurple] (barycentric cs:b1=0.073,b2=0.342,b3=0.512,b4=0.073) circle (2pt);
            \draw[fill,reddishPurple] (barycentric cs:b1=0.061,b2=0.274,b3=0.391,b4=0.274) circle (2pt);
            \draw[fill,reddishPurple] (barycentric cs:b1=0.342,b2=0.073,b3=0.512,b4=0.073) circle (2pt);
            \draw[fill,reddishPurple] (barycentric cs:b1=0.274,b2=0.061,b3=0.391,b4=0.274) circle (2pt);
            \draw[fill,reddishPurple] (barycentric cs:b1=0.274,b2=0.274,b3=0.391,b4=0.061) circle (2pt);
            \draw[fill,reddishPurple] (barycentric cs:b1=0.228,b2=0.228,b3=0.316,b4=0.228) circle (2pt);
            \draw[fill,skyBlue] (barycentric cs:b1=0.092,b2=0.092,b3=0.092,b4=0.725) circle (2pt);
            \draw[fill,skyBlue] (barycentric cs:b1=0.073,b2=0.342,b3=0.073,b4=0.512) circle (2pt);
            \draw[fill,skyBlue] (barycentric cs:b1=0.073,b2=0.073,b3=0.342,b4=0.512) circle (2pt);
            \draw[fill,skyBlue] (barycentric cs:b1=0.061,b2=0.274,b3=0.274,b4=0.391) circle (2pt);
            \draw[fill,skyBlue] (barycentric cs:b1=0.342,b2=0.073,b3=0.073,b4=0.512) circle (2pt);
            \draw[fill,skyBlue] (barycentric cs:b1=0.274,b2=0.274,b3=0.061,b4=0.391) circle (2pt);
            \draw[fill,skyBlue] (barycentric cs:b1=0.274,b2=0.061,b3=0.274,b4=0.391) circle (2pt);
            \draw[fill,skyBlue] (barycentric cs:b1=0.228,b2=0.228,b3=0.228,b4=0.316) circle (2pt);
            \draw[fill,red] (barycentric cs:b1=0.725,b2=0.092,b3=0.092,b4=0.092) circle (2pt);
            \draw[fill,red] (barycentric cs:b1=0.512,b2=0.073,b3=0.073,b4=0.342) circle (2pt);
            \draw[fill,red] (barycentric cs:b1=0.512,b2=0.073,b3=0.342,b4=0.073) circle (2pt);
            \draw[fill,red] (barycentric cs:b1=0.391,b2=0.061,b3=0.274,b4=0.274) circle (2pt);
            \draw[fill,red] (barycentric cs:b1=0.512,b2=0.342,b3=0.073,b4=0.073) circle (2pt);
            \draw[fill,red] (barycentric cs:b1=0.391,b2=0.274,b3=0.061,b4=0.274) circle (2pt);
            \draw[fill,red] (barycentric cs:b1=0.391,b2=0.274,b3=0.274,b4=0.061) circle (2pt);
            \draw[fill,red] (barycentric cs:b1=0.316,b2=0.228,b3=0.228,b4=0.228) circle (2pt);
        \end{scope}
    \end{tikzpicture}
\end{Figure} % https://www.desmos.com/geometry-beta/4ozdl0idri

%\subsubsection{Integration Strategy by }

\subsection{Subdivision Integration Strategy}

\subsubsection{Edge Subdivision}
\label{sec:subdivision3Dedge}

\begin{Figure}
    \centering
    \begin{tikzpicture}
        \begin{scope}[engineering axonometry]
            \draw[fill] (0,0,0) +(0,0,-1.4) node {Simplex with characteristic points};
            \begin{scope}[scale=1.7]
                \coordinate (b1) at ({sqrt(8/9)},0,0);
                \coordinate (b2) at ({-sqrt(2/9)},{sqrt(2/3)},0);
                \coordinate (b3) at ({-sqrt(2/9)},{-sqrt(2/3)},0);
                \coordinate (b4) at (0,0,4/3);
            \end{scope}
            \draw (b1) -- (b2);
            \draw (b2) -- (b3);
            \draw (b3) -- (b1);
            \draw (b1) -- (b4);
            \draw (b2) -- (b4);
            \draw (b3) -- (b4);
            \node[below] at (b1) {$B_1$};
            \node[right] at (b2) {$B_2$};
            \node[left]  at (b3) {$B_3$};
            \node[above] at (b4) {$B_4$};
            %% Centroid
            \draw[fill] (barycentric cs:b1=0.25,b2=0.25,b3=0.25,b4=0.25) coordinate (bc) circle (2pt) node[right] {$B_{15}$};
            %% Center of Planes
            %\draw[fill] (barycentric cs:b1=0.33,b2=0.33,b3=0.33,b4=0.00) coordinate (ba4) circle (2pt);% node[below right] {$B_1$};
            %\draw[fill] (barycentric cs:b1=0.00,b2=0.33,b3=0.33,b4=0.33) coordinate (ba1) circle (2pt);% node[below right] {$B_2$};
            %\draw[fill] (barycentric cs:b1=0.33,b2=0.00,b3=0.33,b4=0.33) coordinate (ba2) circle (2pt);% node[below right] {$B_3$};
            %\draw[fill] (barycentric cs:b1=0.33,b2=0.33,b3=0.00,b4=0.33) coordinate (ba3) circle (2pt);% node[below right] {$B_4$};
            %% Center of Edges
            \draw[fill] (barycentric cs:b1=0.50,b2=0.50,b3=0.00,b4=0.00) coordinate (be12) circle (2pt) node[below] {$B_{5}$};
            \draw[fill] (barycentric cs:b1=0.00,b2=0.50,b3=0.50,b4=0.00) coordinate (be23) circle (2pt) node[below] {$B_{6}$};
            \draw[fill] (barycentric cs:b1=0.50,b2=0.00,b3=0.50,b4=0.00) coordinate (be13) circle (2pt) node[below left] {$B_{7}$};
            \draw[fill] (barycentric cs:b1=0.50,b2=0.00,b3=0.00,b4=0.50) coordinate (be14) circle (2pt) node[above right] {$B_{8}$};
            \draw[fill] (barycentric cs:b1=0.00,b2=0.50,b3=0.00,b4=0.50) coordinate (be24) circle (2pt) node[above right] {$B_{9}$};
            \draw[fill] (barycentric cs:b1=0.00,b2=0.00,b3=0.50,b4=0.50) coordinate (be34) circle (2pt) node[above left] {$B_{10}$};
        \end{scope}
        \begin{scope}[shift={(4.5,0)},engineering axonometry]
            \draw[fill] (0,0,0) +(0,0,-1.4) node {Child Simplices};
            \begin{scope}[scale=1.7]
                \coordinate (b1) at ({sqrt(8/9)},0,0);
                \coordinate (b2) at ({-sqrt(2/9)},{sqrt(2/3)},0);
                \coordinate (b3) at ({-sqrt(2/9)},{-sqrt(2/3)},0);
                \coordinate (b4) at (0,0,4/3);
            \end{scope}
            %\draw (b1) -- (b2);
            %\draw (b2) -- (b3);
            %\draw (b3) -- (b1);
            %\draw (b1) -- (b4);
            %\draw (b2) -- (b4);
            %\draw (b3) -- (b4);
            \node[below] at (b1) {$B_1$};
            \node[right] at (b2) {$B_2$};
            \node[left]  at (b3) {$B_3$};
            \node[above] at (b4) {$B_4$};
            %% Centroid
            %\draw[fill] (barycentric cs:b1=0.25,b2=0.25,b3=0.25,b4=0.25) coordinate (b15) circle (2pt);% node[right] {$B_{15}$};
            %% Center of Planes
            %\draw[fill] (barycentric cs:b1=0.33,b2=0.33,b3=0.33,b4=0.00) coordinate (ba4) circle (2pt);% node[below right] {$B_1$};
            %\draw[fill] (barycentric cs:b1=0.00,b2=0.33,b3=0.33,b4=0.33) coordinate (ba1) circle (2pt);% node[below right] {$B_2$};
            %\draw[fill] (barycentric cs:b1=0.33,b2=0.00,b3=0.33,b4=0.33) coordinate (ba2) circle (2pt);% node[below right] {$B_3$};
            %\draw[fill] (barycentric cs:b1=0.33,b2=0.33,b3=0.00,b4=0.33) coordinate (ba3) circle (2pt);% node[below right] {$B_4$};
            %% Center of Edges
            \draw[fill] (barycentric cs:b1=0.50,b2=0.50,b3=0.00,b4=0.00) coordinate (be5) circle (2pt);% node[below] {$B_{5}$};
            \draw[fill] (barycentric cs:b1=0.00,b2=0.50,b3=0.50,b4=0.00) coordinate (be6) circle (2pt);% node[below] {$B_{6}$};
            \draw[fill] (barycentric cs:b1=0.50,b2=0.00,b3=0.50,b4=0.00) coordinate (be7) circle (2pt);% node[below left] {$B_{7}$};
            \draw[fill] (barycentric cs:b1=0.50,b2=0.00,b3=0.00,b4=0.50) coordinate (be8) circle (2pt);% node[above right] {$B_{8}$};
            \draw[fill] (barycentric cs:b1=0.00,b2=0.50,b3=0.00,b4=0.50) coordinate (be9) circle (2pt);% node[above right] {$B_{9}$};
            \draw[fill] (barycentric cs:b1=0.00,b2=0.00,b3=0.50,b4=0.50) coordinate (be10) circle (2pt);% node[above left] {$B_{10}$};
            \newcommand{\drawsimplex}[4]{
                \draw[semithick,blue,line join=round] (#1) -- (#2);
                \draw[semithick,blue,line join=round] (#1) -- (#3);
                \draw[semithick,blue,line join=round] (#1) -- (#4);
                \draw[semithick,blue,line join=round] (#2) -- (#3);
                \draw[semithick,blue,line join=round] (#2) -- (#4);
                \draw[semithick,blue,line join=round] (#3) -- (#4);
            }
            \drawsimplex{b1}{be5}{be8}{be7}
            \drawsimplex{b2}{be5}{be6}{be9}
            \drawsimplex{b3}{be6}{be7}{be10}
            \drawsimplex{b4}{be8}{be9}{be10}
        \end{scope}
        \begin{scope}[shift={(9,0)},engineering axonometry]
            \draw[fill] (0,0,0) +(0,0,-1.4) node {Child Octahedron};
            \begin{scope}[scale=1.7]
                \coordinate (b1) at ({sqrt(8/9)},0,0);
                \coordinate (b2) at ({-sqrt(2/9)},{sqrt(2/3)},0);
                \coordinate (b3) at ({-sqrt(2/9)},{-sqrt(2/3)},0);
                \coordinate (b4) at (0,0,4/3);
            \end{scope}
            %\draw (b1) -- (b2);
            %\draw (b2) -- (b3);
            %\draw (b3) -- (b1);
            %\draw (b1) -- (b4);
            %\draw (b2) -- (b4);
            %\draw (b3) -- (b4);
            %\node[below] at (b1) {$B_1$};
            %\node[right] at (b2) {$B_2$};
            %\node[left]  at (b3) {$B_3$};
            %\node[above] at (b4) {$B_4$};
            %% Centroid
            %\draw[fill] (barycentric cs:b1=0.25,b2=0.25,b3=0.25,b4=0.25) coordinate (b15) circle (2pt);% node[right] {$B_{15}$};
            %% Center of Planes
            %\draw[fill] (barycentric cs:b1=0.33,b2=0.33,b3=0.33,b4=0.00) coordinate (ba4) circle (2pt);% node[below right] {$B_1$};
            %\draw[fill] (barycentric cs:b1=0.00,b2=0.33,b3=0.33,b4=0.33) coordinate (ba1) circle (2pt);% node[below right] {$B_2$};
            %\draw[fill] (barycentric cs:b1=0.33,b2=0.00,b3=0.33,b4=0.33) coordinate (ba2) circle (2pt);% node[below right] {$B_3$};
            %\draw[fill] (barycentric cs:b1=0.33,b2=0.33,b3=0.00,b4=0.33) coordinate (ba3) circle (2pt);% node[below right] {$B_4$};
            %% Center of Edges
            \draw[fill] (barycentric cs:b1=0.50,b2=0.50,b3=0.00,b4=0.00) coordinate (be5) circle (2pt) node[below] {$B_{5}$};
            \draw[fill] (barycentric cs:b1=0.00,b2=0.50,b3=0.50,b4=0.00) coordinate (be6) circle (2pt) node[right=1em] {$B_{6}$};
            \draw[fill] (barycentric cs:b1=0.50,b2=0.00,b3=0.50,b4=0.00) coordinate (be7) circle (2pt) node[below left] {$B_{7}$};
            \draw[fill] (barycentric cs:b1=0.50,b2=0.00,b3=0.00,b4=0.50) coordinate (be8) circle (2pt) node[left=1em] {$B_{8}$};
            \draw[fill] (barycentric cs:b1=0.00,b2=0.50,b3=0.00,b4=0.50) coordinate (be9) circle (2pt) node[above right] {$B_{9}$};
            \draw[fill] (barycentric cs:b1=0.00,b2=0.00,b3=0.50,b4=0.50) coordinate (be10) circle (2pt) node[above left] {$B_{10}$};
            \newcommand{\drawtriangle}[3]{
                \draw[semithick,blue,line join=round] (#1) -- (#2) -- (#3) -- cycle;
            }
            \drawtriangle{be5}{be9}{be8}
            \drawtriangle{be5}{be7}{be6}
            \drawtriangle{be6}{be9}{be10}
            \drawtriangle{be8}{be7}{be10}
        \end{scope}
    \end{tikzpicture}
\end{Figure}

The smooth subdivision scheme proposed in \cite{schaefer_smooth_2004} can be adapted to serve a uniform decomposition of a unit tetrahedron.
This approach circumvents a directional bias in the subdivision scheme.

The points being \begin{align}
    && B_1 &= \begin{bmatrix}
        1 \\
        0 \\
        0 \\
        0 \\
    \end{bmatrix} & B_2 &= \begin{bmatrix}
        0 \\
        1 \\
        0 \\
        0 \\
    \end{bmatrix} & B_3 &= \begin{bmatrix}
        0 \\
        0 \\
        1 \\
        0 \\
    \end{bmatrix} & B_4 &= \begin{bmatrix}
        0 \\
        0 \\
        0 \\
        1 \\
    \end{bmatrix} \\
    B_5 &= \begin{bmatrix}
        \frac{1}{2} \\
        \frac{1}{2} \\
        0 \\
        0 \\
    \end{bmatrix} & B_6 &= \begin{bmatrix}
        0 \\
        \frac{1}{2} \\
        \frac{1}{2} \\
        0 \\
    \end{bmatrix} & B_7 &= \begin{bmatrix}
        \frac{1}{2} \\
        0 \\
        \frac{1}{2} \\
        0 \\
    \end{bmatrix} & B_8 &= \begin{bmatrix}
        \frac{1}{2} \\
        0 \\
        0 \\
        \frac{1}{2} \\
    \end{bmatrix} & B_9 &= \begin{bmatrix}
        0 \\
        \frac{1}{2} \\
        0 \\
        \frac{1}{2} \\
    \end{bmatrix} &B_{10} &= \begin{bmatrix}
        0 \\
        0 \\
        \frac{1}{2} \\
        \frac{1}{2} \\
    \end{bmatrix} \\
    && B_{11} &= \begin{bmatrix}
        0 \\
        \frac{1}{3} \\
        \frac{1}{3} \\
        \frac{1}{3} \\
    \end{bmatrix} & B_{12} &= \begin{bmatrix}
        \frac{1}{3} \\
        0 \\
        \frac{1}{3} \\
        \frac{1}{3} \\
    \end{bmatrix} & B_{13} &= \begin{bmatrix}
        \frac{1}{3} \\
        \frac{1}{3} \\
        0 \\
        \frac{1}{3} \\
    \end{bmatrix} & B_{14} &= \begin{bmatrix}
        \frac{1}{3} \\
        \frac{1}{3} \\
        \frac{1}{3} \\
        0 \\
    \end{bmatrix} & B_{15} &= \begin{bmatrix}
        \frac{1}{4} \\
        \frac{1}{4} \\
        \frac{1}{4} \\
        \frac{1}{4} \\
    \end{bmatrix}
\end{align}

The tetrahedral integration Domains can be expressed as \begin{align}
    \simplex_1 &= \begin{bmatrix}
        B_1 & B_5 & B_7 & B_8 \\
    \end{bmatrix} & \simplex_2 &= \begin{bmatrix}
        B_5 & B_2 & B_6 & B_9 \\
    \end{bmatrix} \\
    \simplex_3 &= \begin{bmatrix}
        B_7 & B_6 & B_3 & B_{10} \\
    \end{bmatrix} & \simplex_4 &= \begin{bmatrix}
        B_8 & B_9 & B_{10} & B_4 \\
    \end{bmatrix}
\end{align}
Whereas the vertices of the octahedral domain are expressable as \begin{align}
    \mathbf{O}_5 &= \begin{bmatrix}
        B_{5} & B_6 & B_7 & B_{8} & B_9 & B_{10} \\
    \end{bmatrix}
\end{align}\marginnote{This order is necessary to uphold. Every even permutation of the order is also okay. The first and last coordinates must be opposing.}
This map implies a order of the octahedrons vertices, which is now necessarily a condition to uphold.

\paragraph{octahedron subdivision}

The subdivision will subdivide the octahedron into $6$ octahedrons and $8$ tetrahedrons, where the $6$ octahedrons will be 
again subdivided into $4$ deformed tetrahedrons. This results in $32$ tetrahedrons of equal volume.
For ease of implementation and reduced complexicity, the subregions of the octahedron will be expressed in \textbf{generalized barycentric coordinates}.
Unfortunately these barycentric coordinates are not unique for any given point inside the octahedron.
Fortunately these coordinates only serve the purpose of interpolation of points inside any octahedron, where we can choose the coordinates of the octahedron
in such a way, that the minimal number of points required is used.

\begin{Figure}
    \centering
    \begin{tikzpicture}
        \begin{scope}[shift={(0,0)},engineering axonometry]
            \draw[fill] (0,0,0) +(0,0,-1.4) node {Octahedron $\mathbf{O}$};
            \begin{scope}[scale=5.0]
                \coordinate (b1) at ({sqrt(8/9)},0,0);
                \coordinate (b2) at ({-sqrt(2/9)},{sqrt(2/3)},0);
                \coordinate (b3) at ({-sqrt(2/9)},{-sqrt(2/3)},0);
                \coordinate (b4) at (0,0,4/3);
            \end{scope}
            \draw[fill] (barycentric cs:b1=0.50,b2=0.50,b3=0.00,b4=0.00) coordinate (o1) circle (2pt) node[below]       {$O_{1}$};
            \draw[fill] (barycentric cs:b1=0.00,b2=0.50,b3=0.50,b4=0.00) coordinate (o2) circle (2pt) node[right]   {$O_{2}$};
            \draw[fill] (barycentric cs:b1=0.50,b2=0.00,b3=0.50,b4=0.00) coordinate (o3) circle (2pt) node[below left]  {$O_{3}$};
            \draw[fill] (barycentric cs:b1=0.50,b2=0.00,b3=0.00,b4=0.50) coordinate (o4) circle (2pt) node[left]    {$O_{4}$};
            \draw[fill] (barycentric cs:b1=0.00,b2=0.50,b3=0.00,b4=0.50) coordinate (o5) circle (2pt) node[above right] {$O_{5}$};
            \draw[fill] (barycentric cs:b1=0.00,b2=0.00,b3=0.50,b4=0.50) coordinate (o6) circle (2pt) node[above left]  {$O_{6}$};
            %
            \draw[fill] (barycentric cs:o1=0.5,o3=0.5) coordinate (o7) circle (2pt) node[above] {$O_{7}$};
            \draw[fill] (barycentric cs:o3=0.5,o4=0.5) coordinate (o8) circle (2pt) node[right] {$O_{8}$};
            \draw[fill] (barycentric cs:o1=0.5,o4=0.5) coordinate (o9) circle (2pt) node[left]  {$O_{9}$};
            \draw[fill] (barycentric cs:o4=0.5,o5=0.5) coordinate (o10) circle (2pt) node[above] {$O_{10}$};
            \draw[fill] (barycentric cs:o5=0.5,o1=0.5) coordinate (o11) circle (2pt) node[right] {$O_{11}$};
            \draw[fill] (barycentric cs:o5=0.5,o2=0.5) coordinate (o12) circle (2pt) node[left] {$O_{12}$};
            \draw[fill] (barycentric cs:o1=0.5,o2=0.5) coordinate (o13) circle (2pt) node[below left] {$O_{13}$};
            \draw[fill] (barycentric cs:o3=0.5,o2=0.5) coordinate (o14) circle (2pt) node[above] {$O_{14}$};
            \draw[fill] (barycentric cs:o3=0.5,o6=0.5) coordinate (o15) circle (2pt) node[above left] {$O_{15}$};
            \draw[fill] (barycentric cs:o4=0.5,o6=0.5) coordinate (o16) circle (2pt) node[above right] {$O_{16}$};
            \draw[fill] (barycentric cs:o5=0.5,o6=0.5) coordinate (o17) circle (2pt) node[above] {$O_{17}$};
            \draw[fill] (barycentric cs:o2=0.5,o6=0.5) coordinate (o18) circle (2pt) node[right] {$O_{18}$};
            \draw[fill] (barycentric cs:o1=0.5,o6=0.5) coordinate (o19) circle (2pt) node[right] {$O_{19}$};
            \newcommand{\drawtriangle}[4][blue]{
                \draw[semithick,#1,line join=round] (#2) -- (#3) -- (#4) -- cycle;
            }
            \drawtriangle{o1}{o5}{o4}
            \drawtriangle{o1}{o3}{o2}
            \drawtriangle{o2}{o5}{o6}
            \drawtriangle{o4}{o3}{o6}
        \end{scope}
        \begin{scope}[shift={(-3,-6)},engineering axonometry]
            \draw[fill] (0,0,0) +(0,0,-1.4) node {Child tetrahedrons $\simplex_{O,i}$};
            \begin{scope}[scale=5.0]
                \coordinate (b1) at ({sqrt(8/9)},0,0);
                \coordinate (b2) at ({-sqrt(2/9)},{sqrt(2/3)},0);
                \coordinate (b3) at ({-sqrt(2/9)},{-sqrt(2/3)},0);
                \coordinate (b4) at (0,0,4/3);
            \end{scope}
            \draw[fill] (barycentric cs:b1=0.50,b2=0.50,b3=0.00,b4=0.00) coordinate (o1);% circle (2pt);% node[below]       {$O_{1}$};
            \draw[fill] (barycentric cs:b1=0.00,b2=0.50,b3=0.50,b4=0.00) coordinate (o2);% circle (2pt);% node[right]   {$O_{2}$};
            \draw[fill] (barycentric cs:b1=0.50,b2=0.00,b3=0.50,b4=0.00) coordinate (o3);% circle (2pt);% node[below left]  {$O_{3}$};
            \draw[fill] (barycentric cs:b1=0.50,b2=0.00,b3=0.00,b4=0.50) coordinate (o4);% circle (2pt);% node[left]    {$O_{4}$};
            \draw[fill] (barycentric cs:b1=0.00,b2=0.50,b3=0.00,b4=0.50) coordinate (o5);% circle (2pt);% node[above right] {$O_{5}$};
            \draw[fill] (barycentric cs:b1=0.00,b2=0.00,b3=0.50,b4=0.50) coordinate (o6);% circle (2pt);% node[above left]  {$O_{6}$};
            %
            \draw[fill] (barycentric cs:o1=0.5,o3=0.5) coordinate (o7) ; %circle (2pt) ;%node[above] {$O_{7}$};
            \draw[fill] (barycentric cs:o3=0.5,o4=0.5) coordinate (o8) ; %circle (2pt) ;%node[right] {$O_{8}$};
            \draw[fill] (barycentric cs:o1=0.5,o4=0.5) coordinate (o9) ; %circle (2pt) ;%node[left]  {$O_{9}$};
            \draw[fill] (barycentric cs:o4=0.5,o5=0.5) coordinate (o10); %circle (2pt);% node[above] {$O_{10}$};
            \draw[fill] (barycentric cs:o5=0.5,o1=0.5) coordinate (o11); %circle (2pt);% node[right] {$O_{11}$};
            \draw[fill] (barycentric cs:o5=0.5,o2=0.5) coordinate (o12); %circle (2pt);% node[left] {$O_{12}$};
            \draw[fill] (barycentric cs:o1=0.5,o2=0.5) coordinate (o13); %circle (2pt);% node[below left] {$O_{13}$};
            \draw[fill] (barycentric cs:o3=0.5,o2=0.5) coordinate (o14); %circle (2pt);% node[above] {$O_{14}$};
            \draw[fill] (barycentric cs:o3=0.5,o6=0.5) coordinate (o15); %circle (2pt);% node[above left] {$O_{15}$};
            \draw[fill] (barycentric cs:o4=0.5,o6=0.5) coordinate (o16); %circle (2pt);% node[above right] {$O_{16}$};
            \draw[fill] (barycentric cs:o5=0.5,o6=0.5) coordinate (o17); %circle (2pt);% node[above] {$O_{17}$};
            \draw[fill] (barycentric cs:o1=0.5,o6=0.5) coordinate (o19); %circle (2pt);% node[right] {$O_{19}$};
            \draw[fill] (barycentric cs:o2=0.5,o6=0.5) coordinate (o18); %circle (2pt);% node[right] {$O_{18}$};
            \draw[fill] (barycentric cs:o1=0.5,o6=0.5) coordinate (o19); %circle (2pt);
            \newcommand{\drawtriangle}[4][blue]{
                \draw[semithick,#1,line join=round] (#2) -- (#3) -- (#4) -- cycle;
            }
            \newcommand{\drawsimplex}[5][blue]{
                \draw[semithick,#1,line join=round] (#2) -- (#3);
                \draw[semithick,#1,line join=round] (#2) -- (#4);
                \draw[semithick,#1,line join=round] (#2) -- (#5);
                \draw[semithick,#1,line join=round] (#3) -- (#4);
                \draw[semithick,#1,line join=round] (#3) -- (#5);
                \draw[semithick,#1,line join=round] (#4) -- (#5);
            }
            \newcommand{\transcoords}[3]{
                \foreach \p in {#1} {
                    \path[vermillion] (#2) -- +($(\p)-(#3)$) coordinate (t\p);
                };
            }
            \newcommand{\explodeview}[6][blue]{
                \path (#5) -- (barycentric cs:#2=0.5,#3=0.5,#4=0.5,#5=0.5) coordinate (t0) circle (1pt);
                \path ($(#5)!#6!(t0)$) coordinate (t01);
                \transcoords{#2,#3,#4,#5}{t01}{#5}
                \drawsimplex[#1]{t#2}{t#3}{t#4}{t#5}
            }
            \path (o19) -- +(1.0,0,0);
            \path (o19) -- +(0.0,1.0,0);
            \path (o19) -- +(0.0,0,1.0);
            %\explodeview{o7}{o13}{o14}{o19}{1ex}
            %\transcoords{o7,o13,o14,o19}{t0}
            \explodeview{o7}{o13}{o14}{o19} {0.3}
            \explodeview{o7}{o8}{o9}{o19}   {0.3}
            \explodeview{o9}{o10}{o11}{o19} {0.3}
            \explodeview{o10}{o16}{o17}{o19}{0.3}
            \explodeview{o15}{o16}{o8}{o19} {0.3}
            \explodeview{o14}{o15}{o18}{o19}{0.3}
            \explodeview{o11}{o13}{o12}{o19}{0.3}
            \explodeview{o12}{o17}{o18}{o19}{0.3}
        \end{scope}
        \begin{scope}[shift={(3,-6)},engineering axonometry]
            \draw[fill] (0,0,0) +(0,0,-1.4) node {Child octahedrons $\mathbf{O}_{O,i}$};
            \begin{scope}[scale=5.0]
                \coordinate (b1) at ({sqrt(8/9)},0,0);
                \coordinate (b2) at ({-sqrt(2/9)},{sqrt(2/3)},0);
                \coordinate (b3) at ({-sqrt(2/9)},{-sqrt(2/3)},0);
                \coordinate (b4) at (0,0,4/3);
            \end{scope}
            \draw[fill] (barycentric cs:b1=0.50,b2=0.50,b3=0.00,b4=0.00) coordinate (o1);% circle (2pt);% node[below]       {$O_{1}$};
            \draw[fill] (barycentric cs:b1=0.00,b2=0.50,b3=0.50,b4=0.00) coordinate (o2);% circle (2pt);% node[right]   {$O_{2}$};
            \draw[fill] (barycentric cs:b1=0.50,b2=0.00,b3=0.50,b4=0.00) coordinate (o3);% circle (2pt);% node[below left]  {$O_{3}$};
            \draw[fill] (barycentric cs:b1=0.50,b2=0.00,b3=0.00,b4=0.50) coordinate (o4);% circle (2pt);% node[left]    {$O_{4}$};
            \draw[fill] (barycentric cs:b1=0.00,b2=0.50,b3=0.00,b4=0.50) coordinate (o5);% circle (2pt);% node[above right] {$O_{5}$};
            \draw[fill] (barycentric cs:b1=0.00,b2=0.00,b3=0.50,b4=0.50) coordinate (o6);% circle (2pt);% node[above left]  {$O_{6}$};
            %
            \draw[fill] (barycentric cs:o1=0.5,o3=0.5) coordinate (o7) ; %circle (2pt) ;%node[above] {$O_{7}$};
            \draw[fill] (barycentric cs:o3=0.5,o4=0.5) coordinate (o8) ; %circle (2pt) ;%node[right] {$O_{8}$};
            \draw[fill] (barycentric cs:o1=0.5,o4=0.5) coordinate (o9) ; %circle (2pt) ;%node[left]  {$O_{9}$};
            \draw[fill] (barycentric cs:o4=0.5,o5=0.5) coordinate (o10); %circle (2pt);% node[above] {$O_{10}$};
            \draw[fill] (barycentric cs:o5=0.5,o1=0.5) coordinate (o11); %circle (2pt);% node[right] {$O_{11}$};
            \draw[fill] (barycentric cs:o5=0.5,o2=0.5) coordinate (o12); %circle (2pt);% node[left] {$O_{12}$};
            \draw[fill] (barycentric cs:o1=0.5,o2=0.5) coordinate (o13); %circle (2pt);% node[below left] {$O_{13}$};
            \draw[fill] (barycentric cs:o3=0.5,o2=0.5) coordinate (o14); %circle (2pt);% node[above] {$O_{14}$};
            \draw[fill] (barycentric cs:o3=0.5,o6=0.5) coordinate (o15); %circle (2pt);% node[above left] {$O_{15}$};
            \draw[fill] (barycentric cs:o4=0.5,o6=0.5) coordinate (o16); %circle (2pt);% node[above right] {$O_{16}$};
            \draw[fill] (barycentric cs:o5=0.5,o6=0.5) coordinate (o17); %circle (2pt);% node[above] {$O_{17}$};
            \draw[fill] (barycentric cs:o1=0.5,o6=0.5) coordinate (o19); %circle (2pt);% node[right] {$O_{19}$};
            \draw[fill] (barycentric cs:o2=0.5,o6=0.5) coordinate (o18); %circle (2pt);% node[right] {$O_{18}$};
            \draw[fill] (barycentric cs:o1=0.5,o6=0.5) coordinate (o19); %circle (2pt);
            \newcommand{\drawtriangle}[4][blue]{
                \draw[semithick,#1,line join=round] (#2) -- (#3) -- (#4) -- cycle;
            }
            \newcommand{\drawoctahedron}[7][blue]{
                \draw[semithick,#1,line join=round] (#2) -- (#3) -- (#7) -- (#4) -- cycle;
                \draw[semithick,#1,line join=round] (#2) -- (#5) -- (#7) -- (#6) -- cycle;

                \draw[semithick,#1,line join=round] (#3) -- (#4) -- (#5) -- (#6) --cycle;
            }
            \newcommand{\transcoords}[3]{
                \foreach \p in {#1} {
                    \path[vermillion] (#2) -- +($(\p)-(#3)$) coordinate (t\p);
                };
            }
            \newcommand{\explodeview}[8][blue]{
                \path (#2) -- (barycentric cs:#2=0.5,#3=0.5,#4=0.5,#5=0.5,#6=0.5,#7=0.5) coordinate (t0) circle (1pt);
                \path ($(#2)!#8!(t0)$) coordinate (t01);
                \transcoords{#2,#3,#4,#5,#6,#7}{t01}{#2}
                \drawoctahedron[#1]{t#2}{t#3}{t#4}{t#5}{t#6}{t#7}
            }
            \path (o19) -- +(1.0,0,0);
            \path (o19) -- +(0.0,1.0,0);
            \path (o19) -- +(0.0,0,1.0);
            %\explodeview{o7}{o13}{o14}{o19}{1ex}
            %\transcoords{o7,o13,o14,o19}{t0}
            \explodeview{o19}{o7}{o9}{o11}{o13}{o1}{0.3}
            \explodeview{o19}{o12}{o13}{o14}{o18}{o2}{0.3}
            \explodeview{o19}{o7} {o8} {o15}{o14}{o3}{0.3}
            \explodeview{o19}{o8} {o9} {o10}{o16}{o4}{0.3}
            \explodeview{o19}{o10}{o11}{o12}{o17}{o5}{0.3}
            \explodeview{o19}{o15}{o16}{o17}{o18}{o6}{0.3}
            %\explodeview{o7}{o8}{o9}{o19}   {0.3}
            %\explodeview{o9}{o10}{o11}{o19} {0.3}
            %\explodeview{o10}{o16}{o17}{o19}{0.3}
            %\explodeview{o15}{o16}{o8}{o19} {0.3}
            %\explodeview{o14}{o15}{o18}{o19}{0.3}
            %\explodeview{o11}{o13}{o12}{o19}{0.3}
            %\explodeview{o12}{o17}{o18}{o19}{0.3}
        \end{scope}
    \end{tikzpicture}
\end{Figure}

\begin{align}
    O_1 &= \begin{bmatrix}
        1 \\
        0 \\
        0 \\
        0 \\
        0 \\
        0 \\
    \end{bmatrix} & O_2 &= \begin{bmatrix}
        0 \\
        1 \\
        0 \\
        0 \\
        0 \\
        0 \\
    \end{bmatrix} & O_3 &= \begin{bmatrix}
        0 \\
        0 \\
        1 \\
        0 \\
        0 \\
        0 \\
    \end{bmatrix} & O_4 &= \begin{bmatrix}
        0 \\
        0 \\
        0 \\
        1 \\
        0 \\
        0 \\
    \end{bmatrix} & O_5 &= \begin{bmatrix}
        0 \\
        0 \\
        0 \\
        0 \\
        1 \\
        0 \\
    \end{bmatrix} & O_6 &= \begin{bmatrix}
        0 \\
        0 \\
        0 \\
        0 \\
        0 \\
        1 \\
    \end{bmatrix} \\
    O_7 &= \begin{bmatrix}
        0.5 \\
        0 \\
        0.5 \\
        0 \\
        0 \\
        0 \\
    \end{bmatrix} & 
    O_8 &= \begin{bmatrix}
        0 \\
        0 \\
        0.5 \\
        0.5 \\
        0 \\
        0 \\
    \end{bmatrix} & 
    O_9 &= \begin{bmatrix}
        0.5 \\
        0 \\
        0 \\
        0.5 \\
        0 \\
        0 \\
    \end{bmatrix} & 
    O_{10} &= \begin{bmatrix}
        0 \\
        0 \\
        0 \\
        0.5 \\
        0.5 \\
        0 \\
    \end{bmatrix} & 
    O_{11} &= \begin{bmatrix}
        0.5 \\
        0 \\
        0 \\
        0 \\
        0.5 \\
        0 \\
    \end{bmatrix} & 
    O_{12} &= \begin{bmatrix}
        0 \\
        0.5 \\
        0 \\
        0 \\
        0.5 \\
        0 \\
    \end{bmatrix} \\
    O_{13} &= \begin{bmatrix}
        0.5 \\
        0.5 \\
        0 \\
        0 \\
        0 \\
        0 \\
    \end{bmatrix} & 
    O_{14} &= \begin{bmatrix}
        0 \\
        0.5 \\
        0.5 \\
        0 \\
        0 \\
        0 \\
    \end{bmatrix} & 
    O_{15} &= \begin{bmatrix}
        0 \\
        0 \\
        0.5 \\
        0 \\
        0 \\
        0.5 \\
    \end{bmatrix} & 
    O_{16} &= \begin{bmatrix}
        0 \\
        0 \\
        0 \\
        0.5 \\
        0 \\
        0.5 \\
    \end{bmatrix} & 
    O_{17} &= \begin{bmatrix}
        0 \\
        0 \\
        0 \\
        0 \\
        0.5 \\
        0.5 \\
    \end{bmatrix} & 
    O_{18} &= \begin{bmatrix}
        0 \\
        0.5 \\
        0 \\
        0 \\
        0 \\
        0.5 \\
    \end{bmatrix} \\
    O_{19} &= \begin{bmatrix}
        0.5 \\
        0 \\
        0 \\
        0 \\
        0 \\
        0.5 \\
    \end{bmatrix}
\end{align}

\begin{marginfigure}
    %\newcommand{\drawtriangle}[3]{\draw[semithick,blue,line join=round] (#1) -- (#2) -- (#3) -- cycle;}%
    \centering
    \begin{tikzpicture}
        \begin{scope}[engineering axonometry]
            \begin{scope}[scale=1.7]
                \coordinate (b1) at ({sqrt(8/9)},0,0);
                \coordinate (b2) at ({-sqrt(2/9)},{sqrt(2/3)},0);
                \coordinate (b3) at ({-sqrt(2/9)},{-sqrt(2/3)},0);
                \coordinate (b4) at (0,0,4/3);
            \end{scope}
            \draw[fill] (barycentric cs:b1=0.50,b2=0.50,b3=0.00,b4=0.00) coordinate (be5)  circle (2pt) node[below] {$1$};
            \draw[fill] (barycentric cs:b1=0.00,b2=0.50,b3=0.50,b4=0.00) coordinate (be6)  circle (2pt) node[right=1em] {$2$};
            \draw[fill] (barycentric cs:b1=0.50,b2=0.00,b3=0.50,b4=0.00) coordinate (be7)  circle (2pt) node[below left] {$3$};
            \draw[fill] (barycentric cs:b1=0.50,b2=0.00,b3=0.00,b4=0.50) coordinate (be8)  circle (2pt) node[left=1em] {$4$};
            \draw[fill] (barycentric cs:b1=0.00,b2=0.50,b3=0.00,b4=0.50) coordinate (be9)  circle (2pt) node[above right] {$5$};
            \draw[fill] (barycentric cs:b1=0.00,b2=0.00,b3=0.50,b4=0.50) coordinate (be10) circle (2pt) node[above left] {$6$};
            % 1 -- 2, 2 -- 6, 2 -- 3, 5 -- 2
            % be5 -- be6, be6 -- be10, be6 -- be7, be9 -- be6
            \draw[semithick,blue,line join=round,dashed] (be5) -- (be6);
            \draw[semithick,blue,line join=round] (be5) -- (be7);
            \draw[semithick,blue,line join=round] (be5) -- (be8);
            \draw[semithick,blue,line join=round] (be5) -- (be9);
            %\draw[semithick,blue,line join=round] (be6) -- (be7);
            %\draw[semithick,blue,line join=round] (be7) -- (be8);
            %\draw[semithick,blue,line join=round] (be8) -- (be9);
            %\draw[semithick,blue,line join=round] (be9) -- (be6);
            \draw[semithick,blue,line join=round,dashed] (be10) -- (be6);
            \draw[semithick,blue,line join=round] (be10) -- (be7);
            \draw[semithick,blue,line join=round] (be10) -- (be8);
            \draw[semithick,blue,line join=round] (be10) -- (be9);
            %
            \draw[semithick,dashed,red,line join=round,->-] (be5) -- (be10);
            \draw[semithick,dashed,red,line join=round,->-] (be6) -- (be7);
            \draw[semithick,red,line join=round,->-] (be7) -- (be8);
            \draw[semithick,red,line join=round,->-] (be8) -- (be9);
            \draw[semithick,dashed,red,line join=round,->-] (be9) -- (be6);
            %
            \fill (be5)  circle (2pt);            
            \fill (be6)  circle (2pt);
            \fill (be7)  circle (2pt);
            \fill (be8)  circle (2pt);
            \fill (be9)  circle (2pt);
            \fill (be10) circle (2pt);
        \end{scope}
    \end{tikzpicture}
    \caption{Orientation of Points}
\end{marginfigure}

The child octahedrons are \begin{align}
    \mathbf{O}_{O,1} &= \begin{bmatrix}
        O_1 & O_7 & O_{9} & O_{11} & O_{13} & O_{19} \\
    \end{bmatrix} \\
    \mathbf{O}_{O,2} &= \begin{bmatrix}
        O_{13} & O_{2} & O_{14} & O_{19}& O_{12} & O_{18} \\
    \end{bmatrix} \\
    \mathbf{O}_{O,3} &= \begin{bmatrix}
        O_{7} & O_{3} & O_{8} & O_{19}& O_{14} & O_{15} \\
    \end{bmatrix} \\
    \mathbf{O}_{O,4} &= \begin{bmatrix}
        O_9 & O_{4} & O_{10} & O_{19}& O_{8} & O_{16} \\
    \end{bmatrix} \\
    \mathbf{O}_{O,5} &= \begin{bmatrix}
        O_{11} & O_{5} & O_{12} & O_{19}& O_{10} & O_{17} \\
    \end{bmatrix} \\
    \mathbf{O}_{O,6} &= \begin{bmatrix}
        O_{19} & O_{15} & O_{16} & O_{17}& O_{18} & O_{6} \\
    \end{bmatrix}
\end{align}

The child simplizes are \begin{align}
    \simplex_{O,1} &= \begin{bmatrix}
        O_7 & O_{13} & O_{14} & O_{19} \\
    \end{bmatrix} & \simplex_{O,2} &= \begin{bmatrix}
        O_7 & O_{8} & O_{9} & O_{19} \\
    \end{bmatrix} \\
    \simplex_{O,3} &= \begin{bmatrix}
        O_9 & O_{10} & O_{11} & O_{19} \\
    \end{bmatrix} & \simplex_{O,4} &= \begin{bmatrix}
        O_{10} & O_{16} & O_{17} & O_{19} \\
    \end{bmatrix} \\
    \simplex_{O,5} &= \begin{bmatrix}
        O_{15} & O_{16} & O_{8} & O_{19} \\
    \end{bmatrix} & \simplex_{O,6} &= \begin{bmatrix}
        O_{14} & O_{18} & O_{15} & O_{19} \\
    \end{bmatrix} \\
    \simplex_{O,7} &= \begin{bmatrix}
        O_{11} & O_{13} & O_{12} & O_{19} \\
    \end{bmatrix} & \simplex_{O,8} &= \begin{bmatrix}
        O_{12} & O_{17} & O_{18} & O_{19} \\
    \end{bmatrix}
\end{align}

\paragraph{octahedron integration}

The standard integration will subdivide the octahedron along its diagonal creating $4$ deformed tetrahedrons.
These deformed tetrahedrons are defined by following points: \begin{align}
    \simplex_{O,1} &= \begin{bmatrix}
        O_3 & O_4 & O_1 & O_5 \\
    \end{bmatrix} & \simplex_{O,2} &= \begin{bmatrix}
        O_3 & O_1 & O_2 & O_5 \\
    \end{bmatrix} \\
    \simplex_{O,3} &= \begin{bmatrix}
        O_3 & O_2 & O_6 & O_5 \\
    \end{bmatrix} & \simplex_{O,4} &= \begin{bmatrix}
        O_3 & O_6 & O_4 & O_5 \\
    \end{bmatrix}
\end{align}
The domains $\simplex_{O,i}$ can the be integrated with a pure simplex integrator. This results in $4$ tetrahedrons of equal volume.

\begin{Figure}
    \centering
    \begin{tikzpicture}
        \begin{scope}[shift={(0,0)},engineering axonometry]
            \draw[fill] (0,0,0) +(0,0,-1.4) node {Octahedron $\mathbf{O}$};
            \begin{scope}[scale=5.0]
                \coordinate (b1) at ({sqrt(8/9)},0,0);
                \coordinate (b2) at ({-sqrt(2/9)},{sqrt(2/3)},0);
                \coordinate (b3) at ({-sqrt(2/9)},{-sqrt(2/3)},0);
                \coordinate (b4) at (0,0,4/3);
            \end{scope}
            \draw (b1) -- (b2);
            \draw (b2) -- (b3);
            \draw (b3) -- (b1);
            \draw (b1) -- (b4);
            \draw (b2) -- (b4);
            \draw (b3) -- (b4);
            \draw[fill,red] (barycentric cs:b1=0.469,b2=0.177,b3=0.177,b4=0.177) circle (2pt);
            \draw[fill,red] (barycentric cs:b1=0.177,b2=0.469,b3=0.177,b4=0.177) circle (2pt);
            \draw[fill,red] (barycentric cs:b1=0.177,b2=0.177,b3=0.469,b4=0.177) circle (2pt);
            \draw[fill,red] (barycentric cs:b1=0.177,b2=0.177,b3=0.177,b4=0.469) circle (2pt);
            \draw[fill,red] (barycentric cs:b1=0.734,b2=0.089,b3=0.089,b4=0.089) circle (2pt);
            \draw[fill,red] (barycentric cs:b1=0.589,b2=0.234,b3=0.089,b4=0.089) circle (2pt);
            \draw[fill,red] (barycentric cs:b1=0.589,b2=0.089,b3=0.234,b4=0.089) circle (2pt);
            \draw[fill,red] (barycentric cs:b1=0.589,b2=0.089,b3=0.089,b4=0.234) circle (2pt);
            \draw[fill,red] (barycentric cs:b1=0.234,b2=0.589,b3=0.089,b4=0.089) circle (2pt);
            \draw[fill,red] (barycentric cs:b1=0.089,b2=0.734,b3=0.089,b4=0.089) circle (2pt);
            \draw[fill,red] (barycentric cs:b1=0.089,b2=0.589,b3=0.234,b4=0.089) circle (2pt);
            \draw[fill,red] (barycentric cs:b1=0.089,b2=0.589,b3=0.089,b4=0.234) circle (2pt);
            \draw[fill,red] (barycentric cs:b1=0.323,b2=0.000,b3=0.500,b4=0.177) circle (2pt);
            \draw[fill,red] (barycentric cs:b1=0.323,b2=0.000,b3=0.354,b4=0.323) circle (2pt);
            \draw[fill,red] (barycentric cs:b1=0.177,b2=0.000,b3=0.646,b4=0.177) circle (2pt);
            \draw[fill,red] (barycentric cs:b1=0.177,b2=0.000,b3=0.500,b4=0.323) circle (2pt);
            \draw[fill,red] (barycentric cs:b1=0.234,b2=0.089,b3=0.089,b4=0.589) circle (2pt);
            \draw[fill,red] (barycentric cs:b1=0.089,b2=0.234,b3=0.089,b4=0.589) circle (2pt);
            \draw[fill,red] (barycentric cs:b1=0.089,b2=0.089,b3=0.234,b4=0.589) circle (2pt);
            \draw[fill,red] (barycentric cs:b1=0.089,b2=0.089,b3=0.089,b4=0.734) circle (2pt);
            \draw[fill,red] (barycentric cs:b1=0.411,b2=0.177,b3=0.234,b4=0.177) circle (2pt);
            \draw[fill,red] (barycentric cs:b1=0.411,b2=0.177,b3=0.089,b4=0.323) circle (2pt);
            \draw[fill,red] (barycentric cs:b1=0.411,b2=0.323,b3=0.089,b4=0.177) circle (2pt);
            \draw[fill,red] (barycentric cs:b1=0.266,b2=0.323,b3=0.089,b4=0.323) circle (2pt);
            \draw[fill,red] (barycentric cs:b1=0.323,b2=0.266,b3=0.323,b4=0.089) circle (2pt);
            \draw[fill,red] (barycentric cs:b1=0.323,b2=0.411,b3=0.177,b4=0.089) circle (2pt);
            \draw[fill,red] (barycentric cs:b1=0.177,b2=0.411,b3=0.323,b4=0.089) circle (2pt);
            \draw[fill,red] (barycentric cs:b1=0.177,b2=0.411,b3=0.177,b4=0.234) circle (2pt);
            \draw[fill,red] (barycentric cs:b1=0.234,b2=0.177,b3=0.411,b4=0.177) circle (2pt);
            \draw[fill,red] (barycentric cs:b1=0.089,b2=0.323,b3=0.411,b4=0.177) circle (2pt);
            \draw[fill,red] (barycentric cs:b1=0.089,b2=0.177,b3=0.411,b4=0.323) circle (2pt);
            \draw[fill,red] (barycentric cs:b1=0.089,b2=0.323,b3=0.266,b4=0.323) circle (2pt);
            \draw[fill,red] (barycentric cs:b1=0.323,b2=0.089,b3=0.323,b4=0.266) circle (2pt);
            \draw[fill,red] (barycentric cs:b1=0.177,b2=0.089,b3=0.323,b4=0.411) circle (2pt);
            \draw[fill,red] (barycentric cs:b1=0.323,b2=0.089,b3=0.177,b4=0.411) circle (2pt);
            \draw[fill,red] (barycentric cs:b1=0.177,b2=0.234,b3=0.177,b4=0.411) circle (2pt);
        \end{scope}
    \end{tikzpicture}
\end{Figure}

\pagebreak

\part{Implementation}

\section{2D Case}

Many formulas have been implemented 1-to-1 into code, shifting indizes as needed.

For the hierarchic subdivision algorithm a tree data structure has been utilized.
The depth of the tree corresponds to the number of subdivisions, while the tree nodes visited contain a number.
Every node also contains the information, whether a refinement has been tested. That way, repeated function 
evaluations are circumvented.

The tree nodes are visited via a depth-first traversal. This choice is arbitrary.  Upon visiting a leaf node, it will be refined, if it hasn't been refined before.
If the error of the unrefined leaf and the refined leaf is above the accepted threshold, the refined leaves will remain in the tree.
This procedure will be done until, no leafs have been refined in an iteration.

Therefore any tree encodes a specific subdivision for a simplex.

\subsection{Magic Numbers and other arbitrary decisions}

In the 2D Simplex, the subdomains are indexed as introduced in \ref{sec:subdivision2D}.

With a list of indeces the subdomain transformation can be derived.

\section{3D Case}

\subsection{Magic Numbers and other arbitrary decisions}

In the 3D Simplex, the subdivision scheme needs to index every domain.
As there are two types of domains, a distiction must be made.

The Domains listed in \ref{sec:subdivision3Dedge} will be indexed in the following way: \begin{align*}
    &\begin{array}{@{}*{13}{c}@{}}
        0 & 1 & 2 & 3 & 4 & 5 & 6 & 7 & 8 & 9 & 10 & 11 & 12 \\
         & \simplex_1 & \simplex_2 & \simplex_3 & \simplex_4 
        & \simplex_{O,1} & \simplex_{O,2} & \simplex_{O,3} & \simplex_{O,4} & \simplex_{O,5} & \simplex_{O,6}
        & \simplex_{O,7} & \simplex_{O,8} \\
    \end{array} \\
    &\begin{array}{@{}*{7}{c}@{}}
        13 & 14 & 15 & 16 & 17 & 18 & 19 \\
        \mathbf{O}_5 & \mathbf{O}_{O,1} & \mathbf{O}_{O,2} & \mathbf{O}_{O,3} 
        & \mathbf{O}_{O,4} & \mathbf{O}_{O,5} & \mathbf{O}_{O,6}\\
    \end{array} 
\end{align*}

\pagebreak

\printbibliography

\end{document}