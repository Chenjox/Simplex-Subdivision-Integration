% !TeX root: .\Beleg5.tex
% arara: pdflatex
% arara: pdflatex
\documentclass{mitschrift}
\usepackage[utf8]{inputenc}
\usepackage[T1]{fontenc}
\usepackage[ngerman]{babel}

%\usepackage{isodate}
\usepackage{graphicx}
\usepackage{tikz}

\doctitel{Simplex Integration}
\docuntertitel{By Domain Subdivision}

\begin{document}
\makekopf

\tableofcontents

\pagebreak

\part{Theory}

\section{Problem}

Total Energy in the System: \begin{equation}
    \Pi = \int_\Omega g(\phi) \psi(\boldsymbol{u}) \diff \Omega + \frac{G_c}{2l} \int_\Omega \phi^2 + l^2 \nabla \phi \cdot \nabla \phi \diff \Omega \abbild \min
\end{equation} 

Degradation Function \begin{equation}
    g(\phi) = \left(1 - \phi^2\right)^2 + k
\end{equation} mit \begin{conditions}
    k  & being a small but finite scalar such as 10 6 \\
    G_c & critical energy release rate, material parameter \\
    l  & \emph{width} of phase field \\
\psi (u) & strain energy density function\\
    u & displacement function \\
    \phi & phase field parameter, ansatz function discussed below \\
    \nabla \phi & gradient of phase field parameter \\
\end{conditions}

\begin{equation}
    \vdiff_u \Pi = \int_\Omega g(\phi) \sigma(\boldsymbol{u}) \pardiff{\varepsilon}{\boldsymbol{u}} \vdiff \boldsymbol{u} = 0
\end{equation}

\begin{equation}
    \vdiff_\phi \Pi = \int_\Omega 2(\phi - 1) \vdiff\phi \psi(\boldsymbol{u}) \diff \Omega + \frac{G_c}{l} \int_\Omega \phi \vdiff \phi + l^2 \nabla \phi \cdot \nabla \vdiff \phi \diff \Omega = 0
\end{equation}

\subsection{Ansatz functions}

\begin{equation}
    \boldsymbol{u} = \sum_i N_i \boldsymbol{u}_i + \sum_i N_i F \boldsymbol{a}_i 
\end{equation} mit \begin{conditions}
    N_i & are quadratic lagrange (standard) shape functions for tetrahedrons \\
    \boldsymbol{U}_i = \boldsymbol{u}_i, \boldsymbol{a}_i & are nodal degrees of freedom for displacement function \\
    F & is an enrichment function (sigmoid like, depends on $\phi$, later) \\
\end{conditions}

\begin{equation}
    f_{\text{base}} = \sum_i N_i \phi_i
\end{equation}

\begin{equation}
    \varsigmait = \frac{f_{\text{base}}}{\sqrt[4]{f_{\text{base}}^2 + k_{\text{res}}}}
\end{equation} mit \begin{conditions}
k_\text{reg} & small but finite parameter \\
\end{conditions}

\begin{align}
    \phi &= \exp(-\varsigmait) \\
    \phi &= \exp(-\frac{\varsigmait}{l})
\end{align}

we need to be able to integrate the residual vectors and the stiffness matrices
efficiently and accurately
\begin{equation}
    \vdiff_{\boldsymbol{U}_i} \Pi = \int_\Omega g(\phi)\boldsymbol{\sigma}(\boldsymbol{u})\pardiff{\varepsilon}{\boldsymbol{u}}\cdot \pardiff{\boldsymbol{u}}{\boldsymbol{U}} \diff \Omega \cdot \vdiff \boldsymbol{U}_i
\end{equation}

\begin{equation}
    \vdiff_{\phi_i} \Pi = \int_\Omega 2(\phi - 1) \pardiff{\phi}{\phi_i} \psi(\boldsymbol{u}) \diff \Omega \vdiff\phi_i + \frac{G_c}{l} \int_\Omega \phi \pardiff{\phi}{\phi_i} + l^2 \nabla \phi \cdot \pardiff{\nabla \phi}{\phi_i} \diff \Omega \vdiff \phi_i = 0
\end{equation}

\begin{equation}
    \Delta_{U_i} \vdiff_{U_i} \Pi = U_i \cdot \int_\Omega g(\phi) \pardiff{\varepsilon}{U_i}\cdot \mathbb{C}\cdot \pardiff{\varepsilon}{U_i} \diff \Omega \cdot \vdiff U_i 
\end{equation}

\begin{align}
    \Delta_{\phi_j} \vdiff_{\phi_i} \Pi &= \phi_j \int_\Omega 2\left(\pardiff{\phi}{\phi_i}\right)^2 \psi(u) \diff \Omega \vdiff \phi_i 
    + \phi_j \int_\Omega 2(\phi -1 )\pardiff[^2]{\phi}{\phi_i} \phi(u) \diff \Omega \vdiff \phi_i \\
&\phantom{={}} +\phi_j \frac{G_c}{l} \int_\Omega \pardiff{\phi}{\phi_i} + \pardiff[^2]{\phi}{\phi_i} + l^2 \pardiff{\nabla \phi}{\phi_j} \cdot \pardiff{\nabla \phi}{\phi_i} + l^2 \nabla \phi \cdot \pardiff[^2]{\nabla \phi}{\phi_i^2} \diff \Omega \vdiff \phi_i
\end{align}

\section{Element Description}

\begin{marginfigure}
    \centering
    \begin{tikzpicture}
        \draw[black,thick] (0,0) -- (-60:3cm) -- (-120:3cm) -- cycle;
        \path (0,0) node[above] {$1$};
        \path (-60:3cm) node[above] {$2$};
        \path (-120:3cm) node[above] {$3$};
        \path (-60:1.5cm) node[above] {$5$};
        \path (-120:1.5cm) node[above] {$4$};
        \path (-60:3cm) -- (-120:3cm) node[midway,above] {$6$};
    \end{tikzpicture}
\end{marginfigure}

Shape Functions in barycentric coordinates \begin{align}
    N_1(\xi_1,\xi_2,\xi_3) &= \xi_1 \\
    N_2(\xi_1,\xi_2,\xi_3) &= \xi_2 \\
    N_3(\xi_1,\xi_2,\xi_3) &= \xi_3 \\
    N_4(\xi_1,\xi_2,\xi_3) &= 4 \xi_1\xi_3 \\
    N_5(\xi_1,\xi_2,\xi_3) &= 4 \xi_1\xi_2 \\
    N_6(\xi_1,\xi_2,\xi_3) &= 4 \xi_2\xi_3
\end{align}

Barycentric Interpolation Formula \begin{align}
    P(\xi_1,\xi_2,\xi_3) &= p_1 \xi_1 + p_2 \xi_2 + p_3 \xi_3
\end{align}

$\xi-\eta$-Transformation \begin{align}
    \xi &:= \xi_1 \\
    \eta &:= \xi_2 \\
    \xi_3 &:= 1 - \xi - \eta
\end{align}

\end{document}