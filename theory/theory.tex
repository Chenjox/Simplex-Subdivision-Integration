% !TeX root: .\Beleg5.tex
% arara: clean: { files: [theory.log,theory.aux,theory.thm,theory.slnc] }
% arara: pdflatex
% arara: pdflatex
% arara: pdflatex
\documentclass{mitschrift}
\usepackage[utf8]{inputenc}
\usepackage[T1]{fontenc}
\usepackage[ngerman]{babel}

%\usepackage{isodate}
\usepackage{graphicx}
\usepackage{tikz}

\doctitel{Simplex Integration}
\docuntertitel{By Domain Subdivision}

\newcommand{\simplex}{\boldsymbol{\Delta}}

\begin{document}
\makekopf

\tableofcontents

\pagebreak

\part{Theory}

\section{Problem}

Total Energy in the System: \begin{equation}
    \Pi = \int_\Omega g(\phi) \psi(\boldsymbol{u}) \diff \Omega + \frac{G_c}{2l} \int_\Omega \phi^2 + l^2 \nabla \phi \cdot \nabla \phi \diff \Omega \abbild \min
\end{equation} 

Degradation Function \begin{equation}
    g(\phi) = \left(1 - \phi^2\right)^2 + k
\end{equation} mit \begin{conditions}
    k  & being a small but finite scalar such as 10 6 \\
    G_c & critical energy release rate, material parameter \\
    l  & \emph{width} of phase field \\
\psi (u) & strain energy density function\\
    u & displacement function \\
    \phi & phase field parameter, ansatz function discussed below \\
    \nabla \phi & gradient of phase field parameter \\
\end{conditions}

\begin{equation}
    \vdiff_u \Pi = \int_\Omega g(\phi) \sigma(\boldsymbol{u}) \pardiff{\varepsilon}{\boldsymbol{u}} \vdiff \boldsymbol{u} = 0
\end{equation}

\begin{equation}
    \vdiff_\phi \Pi = \int_\Omega 2(\phi - 1) \vdiff\phi \psi(\boldsymbol{u}) \diff \Omega + \frac{G_c}{l} \int_\Omega \phi \vdiff \phi + l^2 \nabla \phi \cdot \nabla \vdiff \phi \diff \Omega = 0
\end{equation}

\subsection{Ansatz functions}

\begin{equation}
    \boldsymbol{u} = \sum_i N_i \boldsymbol{u}_i + \sum_i N_i F \boldsymbol{a}_i 
\end{equation} mit \begin{conditions}
    N_i & are quadratic lagrange (standard) shape functions for tetrahedrons \\
    \boldsymbol{U}_i = \boldsymbol{u}_i, \boldsymbol{a}_i & are nodal degrees of freedom for displacement function \\
    F & is an enrichment function (sigmoid like, depends on $\phi$, later) \\
\end{conditions}

\begin{equation}
    f_{\text{base}} = \sum_i N_i \phi_i
\end{equation}

\begin{equation}
    \varsigmait = \frac{f_{\text{base}}}{\sqrt[4]{f_{\text{base}}^2 + k_{\text{res}}}}
\end{equation} mit \begin{conditions}
k_\text{reg} & small but finite parameter \\
\end{conditions}

\begin{align}
    \phi &= \exp(-\varsigmait) \\
    \phi &= \exp(-\frac{\varsigmait}{l})
\end{align}

we need to be able to integrate the residual vectors and the stiffness matrices
efficiently and accurately
\begin{equation}
    \vdiff_{\boldsymbol{U}_i} \Pi = \int_\Omega g(\phi)\boldsymbol{\sigma}(\boldsymbol{u})\pardiff{\varepsilon}{\boldsymbol{u}}\cdot \pardiff{\boldsymbol{u}}{\boldsymbol{U}} \diff \Omega \cdot \vdiff \boldsymbol{U}_i
\end{equation}

\begin{equation}
    \vdiff_{\phi_i} \Pi = \int_\Omega 2(\phi - 1) \pardiff{\phi}{\phi_i} \psi(\boldsymbol{u}) \diff \Omega \vdiff\phi_i + \frac{G_c}{l} \int_\Omega \phi \pardiff{\phi}{\phi_i} + l^2 \nabla \phi \cdot \pardiff{\nabla \phi}{\phi_i} \diff \Omega \vdiff \phi_i = 0
\end{equation}

\begin{equation}
    \Delta_{U_i} \vdiff_{U_i} \Pi = U_i \cdot \int_\Omega g(\phi) \pardiff{\varepsilon}{U_i}\cdot \mathbb{C}\cdot \pardiff{\varepsilon}{U_i} \diff \Omega \cdot \vdiff U_i 
\end{equation}

\begin{align}
    \Delta_{\phi_j} \vdiff_{\phi_i} \Pi &= \phi_j \int_\Omega 2\left(\pardiff{\phi}{\phi_i}\right)^2 \psi(u) \diff \Omega \vdiff \phi_i 
    + \phi_j \int_\Omega 2(\phi -1 )\pardiff[^2]{\phi}{\phi_i} \phi(u) \diff \Omega \vdiff \phi_i \\
&\phantom{={}} +\phi_j \frac{G_c}{l} \int_\Omega \pardiff{\phi}{\phi_i} + \pardiff[^2]{\phi}{\phi_i} + l^2 \pardiff{\nabla \phi}{\phi_j} \cdot \pardiff{\nabla \phi}{\phi_i} + l^2 \nabla \phi \cdot \pardiff[^2]{\nabla \phi}{\phi_i^2} \diff \Omega \vdiff \phi_i
\end{align}

\section{Element Description}

\begin{marginfigure}
    \centering
    \begin{tikzpicture}
        \draw[black,thick] (0,0) -- (-60:3cm) -- (-120:3cm) -- cycle;
        \path (0,0) node[above] {$1$};
        \path (-60:3cm) node[below right] {$2$};
        \path (-120:3cm) node[below left] {$3$};
        \path (-60:1.5cm) node[above right] {$5$};
        \path (-120:1.5cm) node[above left] {$4$};
        \path (-60:3cm) -- (-120:3cm) node[midway,below] {$6$};
    \end{tikzpicture}
\end{marginfigure}

Shape Functions in barycentric coordinates \begin{align}
    N_1(\xi_1,\xi_2,\xi_3) &= \xi_1 \\
    N_2(\xi_1,\xi_2,\xi_3) &= \xi_2 \\
    N_3(\xi_1,\xi_2,\xi_3) &= \xi_3 \\
    N_4(\xi_1,\xi_2,\xi_3) &= 4 \xi_1\xi_3 \\
    N_5(\xi_1,\xi_2,\xi_3) &= 4 \xi_1\xi_2 \\
    N_6(\xi_1,\xi_2,\xi_3) &= 4 \xi_2\xi_3
\end{align}

Barycentric Interpolation Formula $P: \mathbb{R}^3 \abbild \mathbb{R}^2$ \begin{align}
    P(\xi_1,\xi_2,\xi_3) &= p_1 \xi_1 + p_2 \xi_2 + p_3 \xi_3
\end{align} mit $p_i \in \mathbb{R}^2$, $\xi_i \in [0,1]$

$\xi$-$\eta$-Transformation \begin{align}
    \xi_1 &:=  1 - \xi - \eta  \\
    \xi_2 &:= \xi\\
    \xi_3 &:=  \eta
\end{align} mit $\xi \in [0,1],\, \eta \in [0,1]$

Es gilt: \begin{align}
    T(\xi, \eta) &= \begin{bmatrix}
        1 - \xi - \eta \\
        \xi \\
        \eta \\
    \end{bmatrix} \\
    T^{-1}(\xi_1,\xi_2,\xi_3) &= \xi_1 \begin{bmatrix}
        0\\
        0\\
    \end{bmatrix} + \xi_2 \begin{bmatrix}
        1\\
        0\\
    \end{bmatrix} + \xi_3 \begin{bmatrix}
        0\\
        1\\
    \end{bmatrix}
\end{align}

\section{Simplex Integration in $n=2$}

\subsection{Simplex Integration}

Characteristic points in barycentric coordinates
\begin{marginfigure}
    \centering
    \begin{tikzpicture}
        \draw[black,thick] (0,0) coordinate (ca) 
          -- (-60:3cm) coordinate (cb) 
          -- (-120:3cm) coordinate (cc) -- cycle;
        %\path (0,0) node[above] {$1$};
        %\path (-60:3cm) node[above right] {$2$};
        %\path (-120:3cm) node[above left] {$3$};
        %\path (-60:1.5cm) node[above right] {$5$};
        %\path (-120:1.5cm) node[above left] {$4$};
        %\path (-60:3cm) -- (-120:3cm) node[midway,above] {$6$};
        \draw[fill] (barycentric cs:ca=0.333,cb=0.333,cc=0.333) coordinate (cbb) circle (2pt) node[below right] {$C_B$};
        \draw[fill] (barycentric cs:ca=1.0,cb=0.0,cc=0.0) coordinate (cb1) circle (2pt) node[above] {$C_1$};
        \draw[fill] (barycentric cs:ca=0.0,cb=1.0,cc=0.0) coordinate (cb2) circle (2pt) node[above right] {$C_2$};
        \draw[fill] (barycentric cs:ca=0.0,cb=0.0,cc=1.0) coordinate (cb3) circle (2pt) node[above left] {$C_3$};
        \draw[fill] (barycentric cs:ca=0.5,cb=0.0,cc=0.5) coordinate (cb4) circle (2pt) node[above left] {$C_4$};
        \draw[fill] (barycentric cs:ca=0.5,cb=0.5,cc=0.0) coordinate (cb5) circle (2pt) node[above right] {$C_5$};
        \draw[fill] (barycentric cs:ca=0.0,cb=0.5,cc=0.5) coordinate (cb6) circle (2pt) node[below] {$C_6$};
        \draw[blue,semithick] (cb1) -- (cb4) -- (cbb) -- (cb5) -- cycle;
        \draw[blue,semithick] (cb2) -- (cb6) -- (cbb) -- (cb5) -- cycle;
        \draw[blue,semithick] (cb3) -- (cb4) -- (cbb) -- (cb6) -- cycle;
        \node at (barycentric cs:cb1=2,cb4=1,cbb=1,cb5=1) {$D_1$};
        \node at (barycentric cs:cb2=2,cb6=1,cbb=1,cb5=1) {$D_2$};
        \node at (barycentric cs:cb3=2,cb4=1,cbb=1,cb6=1) {$D_3$};
    \end{tikzpicture}
\end{marginfigure}
\begin{align}
    C_B &= \begin{bmatrix}
        \frac{1}{3} \\
        \frac{1}{3} \\
        \frac{1}{3} \\
    \end{bmatrix} \\ 
    C_1 &= \begin{bmatrix}
        1 \\
        0 \\
        0 \\
    \end{bmatrix}& C_2 &= \begin{bmatrix}
        0 \\
        1 \\
        0 \\
    \end{bmatrix}& C_3 &= \begin{bmatrix}
        0 \\
        0 \\
        1 \\
    \end{bmatrix} \\
    C_4 &= \begin{bmatrix}
        0.5 \\
        0 \\
        0.5 \\
    \end{bmatrix}& 
    C_5 &= \begin{bmatrix}
        0.5 \\
        0.5 \\
        0 \\
    \end{bmatrix}& 
    C_6 &= \begin{bmatrix}
        0 \\
        0.5 \\
        0.5 \\
    \end{bmatrix}
\end{align}

Domain of a Simplex $\boldsymbol{\Delta}$ can be decomposed into three disjunct subdomains: \begin{equation}
    \boldsymbol{\Delta} = D_1 \cup D_2 \cup D_3
\end{equation}
Therefore the double-Integral \begin{equation}
    \iint_{\boldsymbol{\Delta}} F \diff \boldsymbol{\Delta} = \iint_{D_1} F \diff D_1 + \iint_{D_2} F \diff D_2 + \iint_{D_3} F \diff D_3
\end{equation}

Mapping functions from the $[-1,1] \times [-1,1]$ $X$-$Y$-Unit Square \begin{align}
    g_{1}(X) &= \frac{X}{2} + \frac{1}{2} & g_2(X) &= -\frac{X}{2} + \frac{1}{2} \\
    \pardiff{g_1}{X} &= \frac{1}{2} & \pardiff{g_2}{X} &= -\frac{1}{2} \\
    g_{1}(Y) &= \frac{Y}{2} + \frac{1}{2} & g_2(Y) &= -\frac{Y}{2} + \frac{1}{2} \\
    \pardiff{g_1}{Y} &= \frac{1}{2} & \pardiff{g_2}{Y} &= -\frac{1}{2}
\end{align} to barycentric coordinates of the $D_1,\, D_2,\, D_3$ Quadrilaterials \begin{align}
    C_{D_1}(X,Y) &= C_1 \cdot g_1(X)g_1(Y) + C_5 \cdot g_1(X)g_2(Y) + C_B \cdot g_2(X)g_2(Y) + C_4 \cdot g_2(X)g_1(Y) \\
    C_{D_2}(X,Y) &= C_2 \cdot g_1(X)g_1(Y) + C_6 \cdot g_1(X)g_2(Y) + C_B \cdot g_2(X)g_2(Y) + C_5 \cdot g_2(X)g_1(Y) \\
    C_{D_3}(X,Y) &= C_3 \cdot g_1(X)g_1(Y) + C_4 \cdot g_1(X)g_2(Y) + C_B \cdot g_2(X)g_2(Y) + C_6 \cdot g_2(X)g_1(Y)
\end{align}

\subsection{Integraltransformation}

\pagebreak

\subsection{Simplex Subdivision}

The Integral over a parent Simplex $\boldsymbol{\Delta}_p$, can be expressed as an Integral over 4 Child Simpleces $\boldsymbol{\Delta}_i$: 
\begin{marginfigure}
    \centering
    \begin{tikzpicture}
        \draw[black,thick] (0,0) coordinate (ca) 
          -- (-60:3cm) coordinate (cb) 
          -- (-120:3cm) coordinate (cc) -- cycle;
        %\path (0,0) node[above] {$1$};
        %\path (-60:3cm) node[above right] {$2$};
        %\path (-120:3cm) node[above left] {$3$};
        %\path (-60:1.5cm) node[above right] {$5$};
        %\path (-120:1.5cm) node[above left] {$4$};
        %\path (-60:3cm) -- (-120:3cm) node[midway,above] {$6$};
        %\draw[fill] (barycentric cs:ca=0.333,cb=0.333,cc=0.333) coordinate (cbb) circle (2pt) node[below right] {$C_B$};
        \draw[fill] (barycentric cs:ca=1.0,cb=0.0,cc=0.0) coordinate (cb1) circle (2pt) node[above] {$C_1$};
        \draw[fill] (barycentric cs:ca=0.0,cb=1.0,cc=0.0) coordinate (cb2) circle (2pt) node[above right] {$C_2$};
        \draw[fill] (barycentric cs:ca=0.0,cb=0.0,cc=1.0) coordinate (cb3) circle (2pt) node[above left] {$C_3$};
        \draw[fill] (barycentric cs:ca=0.5,cb=0.0,cc=0.5) coordinate (cb4) circle (2pt) node[above left] {$C_4$};
        \draw[fill] (barycentric cs:ca=0.5,cb=0.5,cc=0.0) coordinate (cb5) circle (2pt) node[above right] {$C_5$};
        \draw[fill] (barycentric cs:ca=0.0,cb=0.5,cc=0.5) coordinate (cb6) circle (2pt) node[below] {$C_6$};
        \draw[blue,semithick] (cb1) -- (cb5) -- (cb4) -- cycle;
        \draw[blue,semithick] (cb2) -- (cb6) -- (cb5) -- cycle;
        \draw[blue,semithick] (cb3) -- (cb4) -- (cb6) -- cycle;
        \draw[blue,semithick] (cb4) -- (cb5) -- (cb6) -- cycle;
        \node at (barycentric cs:cb1=1,cb5=1,cb4=1) {$\boldsymbol{\Delta}_1$};
        \node at (barycentric cs:cb2=1,cb6=1,cb5=1) {$\boldsymbol{\Delta}_2$};
        \node at (barycentric cs:cb3=1,cb4=1,cb6=1) {$\boldsymbol{\Delta}_3$};
        \node at (barycentric cs:cb4=1,cb5=1,cb6=1) {$\boldsymbol{\Delta}_4$};
    \end{tikzpicture}
\end{marginfigure}
\begin{align}
    \iint_{\boldsymbol{\Delta}_p} F \diff\boldsymbol{\Delta}_p &= \iint_{\boldsymbol{\Delta}_1} F \diff\boldsymbol{\Delta}_1 + \iint_{\boldsymbol{\Delta}_2} F \diff\boldsymbol{\Delta}_2 +\iint_{\boldsymbol{\Delta}_3} F \diff\boldsymbol{\Delta}_3 + \iint_{\boldsymbol{\Delta}_4} F \diff\boldsymbol{\Delta}_4
\end{align}

The Coordinates of a Child Simplex $\simplex_i$ can be expressed in local-barycentric coordinates $\xi_{i,1}',\, \xi_{i,2}',\, \xi_{i,3}'$

The corresponding Transformation from the local coordinate System into the global is given by \begin{align}
   T_{lg}(\xi_{i,1}', \xi_{i,2}', \xi_{i,3}') &= C_{i,1}\xi_{i,1}' + C_{i,2}\xi_{i,2}' + C_{i,3} \xi_{i,3}'
\end{align} where $C_{i,j}$ are the coordinates of the Verteces of the Child Simplex $\simplex_i$.



\end{document}